
\mbrpSession{
	title={Všímavost vs. autopilot},
	summary={
		\item „Autopilot“ znamená, že si neuvědomujeme, co děláme. Fungujeme automaticky.
		\item Když máme nutkání či bažení po alkoholu/drogách, často spadneme do autopilota. Pouze reagujeme, bez uvědomění si toho.
		\item Všímavost je opak autopilota. Když jsme všímaví, plně si uvědomujeme, co se v přítomném okamžiku děje.
		\item Všímavost nám pomáhá vyskočit z režimu autopilota a rozhodovat se vědoměji (například v situaci, kdy máme nutkání či bažení).
	},
	formal={\item napojení na sebe; \item \textsc{Brzda}.},
	informal moments={Náhodně během dne se zastavujte a napojujte se na sebe.},
	informal challenges={Používejte \textsc{Brzd}u, když se objeví obtíže.},
	informal activities={Vyzkoušejte všímavé jedení: při každém jídle snězte jedno spousto všímavě. Přineste v tu chvíli plnou pozornost k jídlu a žité zkušenosti jedení.}
}
		
	\pageBRZDA
	
	\clearpage
	\subsection{\textsc{Brzda} s pocitem: bolest a deprese}
	\soberSpace{
		title1=Situace,
		title2={ranní probuzení se silnou fyzickou bolestí a pocitem deprese},
		autopilot={
			thought/{Už je to tu zas. Nezvládám to!},
			action/{Nevstává, zůstává ležet v posteli.},
			thought/{Tahle bolest nikdy nezmizí.},
			thought/{Musím si šlehnout. \emph{(pocit nutkání užít drogu)}},
			action/{Pošle zprávu kamarádovi, aby si opatřil drogu.}
		},
		brzda R={{tlak a bolest v zádech},{smutek},{přichází spousta negativních myšlenek}},
		brzda A={{Připomínám si: \toSelf{Zvládnu to. Už jsem to zvládla i předtím.}},{Vstávám z postele a jdu si dát horkou sprchu.}}
	}

	\clearpage
	\subsection{\textsc{Brzda} s pocitem: nuda a osamělost}
	\soberSpace{
		title1={Situace},
		title2={sobota večer, cítím se znuděně a osaměle},
		autopilot={%
			thought/{Můj život je tak nudný.},
			thought/{Bylo by fajn mít společnost.},
			action/{Volá kamrádovi, se kterým dříve pil a bral drogy.},
			action/{Nutkání k užití.}
		},
		brzda R={{těžkost na hrudníku},{nuda, osamělost},{myšlenky o tom, jak se někomu ozvu.}},
		brzda A={
			{Volám příteli, který mi je oporou v zotavování, naplánujeme si společnou večeři.}
		}
	}

	\pageBRZDAown{\textsc{Brzda} s pocitem: moje}{A teď vy! Popište situaci, která vyvolá automatickou reakci.}{Situace}
	\pageBRZDAown{\textsc{Brzda} s pocitem: moje}{Najděte ještě jeden příklad. Popište jinou situaci, která vyvolá automatickou reakci.}{Situace}

	\pagePracticeLog

\mbrpSession{
	title={Pocity jsou jako hosté},
	summary={
		\item Náročné pocity (úzkost, smutek, vztek) mohou spustit nutkání užít alkohol/drogu. K alkoholu/drogám se můžeme utéci, abychom se vyhnuli náročného pocitu či před ním utekli.
		\item Všímavě vnímat pocity znamená dovolit si tyto pocity v tomto momentě cítit, ne se od nich snažit utéci. Vzít pocity na vědomí (neignorovat je) s otevřeností, zvědavostí a neposuzováním.
		\item Náročné pocity jsou jako „hosté“ či „návštěva“. Navštíví nás a zase odejdou. Jsou dočasné. Náročné pocity mohou být při své návštěvě i nápomocné. \emph{I přesto přivítej každého hosta s úctou. / Možná tě zbaví něčeho, / co ti bránilo prožít nové radosti […] protože ti jej osud posílá / jako učitele.} (báseň \emph{Dům po hosty}).
	},
	formal={\item napojení na sebe; \item všímavost k pocitům (s přečtením si \emph{Domu pro hosty}).},
	informal moments={Náhodně během dne se zastavujte a napojujte se na sebe. Všimněte si pocitů, pokud nějaké vnímáte, s otevřeností a přijetím.},
	informal challenges={Používejte \textsc{Brzd}u, když se objeví obtíže. Cvičte přijímání svých pocitů a dovolte si je cítit.},
	informal activities={Zkoušejte všímavou sprchu či koupel. Přineste v tu chvíli plnou pozornost k tomu, jak se sprchujete či koupete. Napojte se na svých pět smyslů.}
}

	\pageGuesthouse
	\pageBasicFeelings
	\pagePracticeLog

\mbrpSession{
	title={Laskavost k sobě},
	summary={
		\item Sebesoucit (či laskavost k sobě) znamená se k sobě obracet laskavě, jemně a s porozuměním sobě. Sebesoucit nám pomáhá zvládat nezdary či uklouznutí v procesu uzdravování.
		\item Sebesoucit znamená, že se neodsuzujeme, když jsme podráždění či v náročné situaci. Dokážeme se zastavit a říci si: \toSelf{Podrážděnost je v pořádku. Dělám v této situaci nejvíc, co dokážu.}
		\item Všichni lidé zažívají těžké chvilky. Nikdo není dokonalý. Když cvičíme sebesoucit, dokážme si říci: \toSelf{Nejsem jediný, kdo se takto cítí. Všichni to máme těžké.}
	},
	formal={\item napojení na sebe; \item meditace laskavosti.},
	informal moments={Náhodně během dne se zastavujte a napojujte se na sebe.},
	informal challenges={Používejte \textsc{Brzd}u, když se objeví obtíže. Cvičte laskavost k sobě a sebesoucit.},
	informal activities={Zkoušejte všímavé čištění zubů. Přineste v tu chvíli k čištění zubů plnou pozornost. Napojte se na svých pět smyslů.}
}
	\pagePracticeLog

\mbrpSession{
	toc title={Na spouštěče aktivně — ne reaktivně},
	title={Na spouštěče aktivně —\\ ne reaktivně},
	summary={
			\item Spouštěče jsou osoby, místa či věci, které vyvolají nutkání něco udělat či bažení po něčem, např. po droze.
			\item Spouštěče mohou „spustit“ řetěz vjemů, myšlenek a pocitů, ze kterých se skládá celkový prožitek bažení. Zatažení za \textsc{Brzd}u nám pomůže si více uvědomovat samotné spouštěče i vjemy, myšlenky a pocity, které následují.
			\item Zatažení za \textsc{Brzd}u nám pomáhá, když se spouštěč objeví, si uchovat aktivitu a uvědomění, místo automatické reaktivity. Použití \textsc{Brzd}y nám pomáhá rozhodovat se vědomě, z přítomného okamžiku; máme pak méně sklon k užití alkoholu/drogy, když se spouštěč aktivuje.
			\item \textsc{Brzda} je přizpůsobivá a lze za ni zatáhnout různými způsoby. Někdy budete mít čas jen si říct „Brzdi!“ a od spouštěče se vzdálit (např. když vám někdo nabídne drogu). Pak, když už jste odešli, si můžete prosím celou \textsc{Brzd}u, abyste lépe zvládli tuto situaci a bažení, které ještě možná trvá.
	},
	formal={\item napojení na sebe; \item \textsc{Brzda}.},
	informal moments={Náhodně během dne se zastavujte a napojujte se na sebe.},
	informal challenges={Používejte \textsc{Brzd}u, když se objeví obtíže či spouštěče.},
	informal activities={Zkoušejte všímavé poslouchání. Zastavte se a jen vnímejte zvuky z okolí. Zkouštejte to venku i uvnitř. Jinou možností je všímavě poslouchat hudbu. Pusťte si něco, co máte rádi, a změřte pozornost na to, jak vnímáte přicházející zvuky.}
}

	\subsection{\textsc{Brzda} na spouštěč: hádka}
	\soberSpace{
		title1={{\Large\textup{\lightning}} Primární spouštěč},
		title2={hádka s partnerem / partnerkou},
		autopilot={
			action/{Automaticky jdu k nejbližšímu obchodu s alkoholem.},
			action/{\lightning{} \textbf{spouštěč}: \\ Vidím obchod.},
			action/{Jdu dovnitř, koupím si lahev, piju.}
		},
		brzda B={Zpomalte než uděláte cokoliv dalšího.},
		brzda R={
			{těžkost na hrudníku;},
			{hněv, provinilost;},
			{myšlenka \emph{musím se napít}.}
		},
		brzda Z={Zaměřte se na dech. Vnímejte, jak dech proudí dovnitř a ven. Zpomalte ještě o něco víc.},
		brzda D={Rozšiřte znovu svou pozornost na celou situaci. Uvědomte si sebe i své okolí.},
		brzda A={
			{jdu do ložnice, pustím si hudbu;},
			{o něco později si s partnerem/partnerkou promluvím.}
		}
	}

	\clearpage
	\subsection{\textsc{Brzda} na spouštěč: stará známost}
	\soberSpace{
		title1={{\Large\textup{\lightning}} Primární spouštěč},
		title2={přes ulici vidíte starého kamaráda, který vám dříve prodával drogy},
		autopilot={
			action/{Automaticky přejdu ulici ke starému kamarádovi.},
			action/{\lightning{} \textbf{spouštěč}:\\ Řekne mi, že má v autě za rohem drogy.},
			action/{Jdu s ním k jeho autu.},
			action/{Koupím si drogy a dám si.}
		},
		brzda B={Zastavte se a vyskočte z autopilota.},
		brzda R={
			{teplo v oblasti břicha;},
			{bažení či nutkání být high;},
			{myšlenka \vskip0pt plus1cm \toSelf{měl bych ho aspoň jít pozdravit}}
		},
		brzda Z={Pomalu a pozorně se nadechněte a vydechněte.},
		brzda D={Rozšiřte znovu svou pozornost na celou situaci. Uvědomte si sebe i své okolí.},
		brzda A={{Odejdu opačným směrem.},{Když dojdu domů, pustím si z telefonu \textsc{Brzd}u.}}
	}

	\pageBRZDAown{\textsc{Brzda} na spouštěč: vlastní}{A teď vy! Zamyslete se nad \emph{spouštěčem}, který přináší nutkání či bažení po alkoholu či drogách.}{{\Large\textup{\lightning}} Primární spouštěč}
	\pageBRZDAown{\textsc{Brzda} na spouštěč: vlastní}{Najděte ještě jeden příklad \emph{spouštěče}, který přináší nutkání či bažení po alkoholu či drogách.}{{\Large\textup{\lightning}} Primární spouštěč}
	\pagePracticeLog

\mbrpSession{
	title={Vidět myšlenky jako myšlenky},
	summary={
		\item Myšlenky jsou slova či obrazy, které procházejí myslí. Myšlenky se často „vynoří“ samy od sebe. Naše plíce dýchají, naše srdce bije — a právě tak naše mysl myslí. To mysl jednoduše dělá.
		\item Naše myšlenky nejsou nutně pravdivé; nemusíme jim pokaždé věřit či se na ně „nachytat“.
		\item Automaticky věřit vlastním myšlenkám či podle nich jednat (např. \toSelf{Nezvládám to. Kdybych si dal, cítil bych se mnohem lépe.}) může vést k užití alkoholu/drog.
		\item Všímavost nám pomáhá vidět myšlenky jako pouhé myšlenky, ne pravdy či příkazy. Všímavost nám pomáhá udělat „krok zpět“, uvědomit si myšlenky, jak se vynořují, a naložit s nimi aktivně: zdravým způsobem, který nepovede k užití alkoholu/drogy.
	},
	formal={\item napojení na sebe.},
	informal moments={Náhodně během dne se zastavujte a napojujte se na sebe.},
	informal challenges={Používejte \textsc{Brzd}u, když se objeví obtíže.},
	informal activities={Zkoušejte všímavou chůzi. Vneste plnou pozornost do toho, jak jdete. Všímejte si při chůzi všech tělesných vjemů.}
}
	\clearpage
	\subsection{Úloha myšlenek v cyklu relapsu}
		\begin{adjustbox}{width=\linewidth}
		\begin{tikzpicture}[
			every node/.style={
				text centered,
				text width=.35\linewidth,
				draw,
				rounded corners,
				% scale=.85
			},
			node distance=2em and 1em,
			]
			\node(top){\textbf{Spouštěč:}\\fyzická bolest a stres z chátrajícího zdraví};
			\node(initial)[thought,below=of top,callout relative pointer={(0,1)}]{\emph{Prvotní myšlenka:} „Nezvládám to. Bylo by mnohem lepší, kdybych si šlehnul.“};
			\draw[->,color=black!40,line width=1mm] (top)--(initial);

			\node(autopilot)[autopilot,draw=none,below left=of top]{\textbf{\textsc{Autopilot}:}\\věřím myšlenkám,\\jednám podle myšlenek.};
			\node(auto1)[autopilot,below left=of initial]{Věřím myšlenkám. Jsem podrážděný a chci užít drogu. Jdu si opatřit alkohol/drogy.};
			\node(auto2)[autopilot,action,below=of auto1]{Vrátím se domů a dám si jednu tabletku / skleničku / dávku (\toSelf{Jen jednu}).};
			\node(auto3)[autopilot,thought,below=of auto2,callout relative pointer={(0,1)}]{\emph{Myšlenka:}\\ „\textbf{K čertu s tím vším.}“ \\ „Selhal jsem. To už si můžu rovnou dát další!“};
			\node(auto4)[autopilot,action,right=of auto3,yshift=0em]{Piju či si dám další dávku, nakonec usnu.};
			\node(auto5)[autopilot,thought,above=of auto4,callout relative pointer={(0,-1)}]{\emph{Na druhý den…\\ — znova:} \\  „\textbf{K čertu s tím vším}.“ \\ „Věděl jsem, že si zase dám. Jsem beznadějný.“};
			\begin{scope}  % [->,color=red!40,line width=1mm]
				\draw[wide autopilot arrow] (initial.south west) -- (auto1.north east);
				\draw[wide autopilot arrow] (auto1) -- (auto2);
				\draw[wide autopilot arrow] (auto2) -- (auto3);
				\draw[wide autopilot arrow] (auto3) to [bend right](auto4.south);
				\draw[wide autopilot arrow] (auto4) -- (auto5);
				\draw[wide autopilot arrow] (auto5) to [bend right] (auto1.east);
			\end{scope}

			\node(mindful)[mindful,draw=none,below right=of top]{\textbf{\textsc{Všímavost}:}\\ vidět myšlenky jako myšlenky,\\nemusím jednat podle myšlenek.};
			\node(mind1)[mindful,below right=of initial]{Zastavuji se. Uvědomím si myšlenky jen jako myšlenky probíhající v mysli — ne pravdy či příkazy. \toSelf{Tohle je jen myšlenka. Nemusím jí věřit či podle ní hned jednat.}};
			\node(mind2)[mindful,below=of mind1]{Zaměřím na pár chvil pozornost na dech, abych se zakotvil či „uzemnil“.};
			\node(mind3)[mindful,action,below=of mind2]{Akce s uvědoměním a porozuměním. Udělám si čaj. Zavolám někomu, kdo je pro mě oporou.};
			\begin{scope} % [->,ultra thick,color=blue!40,line width=1mm]
				\draw[wide sober arrow] (initial.south east) -- (mind1.north west);
				\draw[wide sober arrow] (mind1) -- (mind2);
				\draw[wide sober arrow] (mind2) -- (mind3);
			\end{scope}
		
			\begin{scope}[->,dashed,line width=1mm,OliveGreen!80]
				\draw (auto3.north east) to [out=85,in=-180,looseness=1.4] 
					(mind1.north west);
				\draw (auto5.north) to [out=90,in=-180]
					node[pos=0.4,solid,draw=none,
						left color=brzdimeAuto!10,right color=brzdimeSober!20,color=black,opacity=.6,text opacity=1,
						text width=.3\linewidth,
						rotate=20,
						yshift=1em,xshift=0em,
					]{\textsc{Nikdy není pozdě se přepnout na všímavost.}}
				(mind1.north west);
			\end{scope}
		\end{tikzpicture}
		\end{adjustbox}
	
	\pagePracticeLog

\mbrpSession{
	title={Jízda na vlně nutkání},
	summary={
		\item Nutkání či bažení jsou jako vlny na vodě. Rostou, dosáhnout vrcholu a zase postupně zmizí.
		\item Všímavost nám pomáhá zpomalit a být zvídavý k této zkušenosti nutkání/bažení. Můžeme se zastavit a zeptat se sami sebe: \toSelf{Hmm, jak toto nutkání právě teď prožívám? Jak v těle? Jak v mysli?}
		\item S všímavostí se můžeme na vlně nutkání „svézt“ — místo přetlačování se s ní, či snahy se ji zbavit. Můžeme tím pěstovat otevřenost a přijímání zkušenosti nutkání, jak se právě děje.
		\item Všmavost nám také může pomoci si více uvědomit, jaké jsou naše vnitřní potřeby a jak o sebe můžeme pečovat, když máme nutkání či bažení. Můžeme zastavit a zeptat se sami sebe: \toSelf{Co v tuto chvíli opravdu potřebuju?}
	},
	formal={\item napojení na sebe.},
	informal moments={Náhodně během dne se zastavujte a napojujte se na sebe.},
	informal challenges={Používejte \textsc{Brzd}u, když se objeví obtíže či nutkání.},
	informal activities={Zkoušejte všímavé dívání se. Zastavte se a otevřete se všem zrakovým vjemům. Všimněte si barev, tvarů, světla a stínů. Zkoušejte to venku. I uvnitř.}
}
	\pagePracticeLog


\mbrpSession{
	title={Následovat své hodnoty},
	summary={
		\item Osobní hodnoty jsou principy a přesvědčení, jak chceme žít svůj život a jací chceme osobně být.
		\item Naše hodnoty jsou kompasem či mapou, podle kterých v životě kráčíme. Naše hodnoty mohou mít vliv na velká i malá rozhodnutí, která během života děláme.
		\item Když jsme na autopilotovi, můžeme se chovat či reagovat v nesouladu s vlastními hodnotami. S všímavostí si své chování uvědomujeme a jednáme s vlastními hodnotami v souladu.
		\item Zotavování ze závislosti znamená své hodnoty následovat a ve svém životě najít smysl a účel. Dotknout se vlastních hodnot nám může dát sílu dělat, co je pro nás opravdu důležité — i když se objeví stres či nepohoda.
	},
	formal={\item napojení na sebe;\item meditace o hodnotách.},
	informal moments={Náhodně během dne se zastavujte a napojujte se na sebe.},
	informal challenges={Používejte \textsc{Brzd}u, když se objeví obtíže. Zvažte své hodnoty a jak se všímavě rozhodovat, abyste s nimi zůstávali v souladu.},
	informal activities={Cvičte všímavost při běžných domácích pracech, např. mytí nádobí či skládání prádla. Vneste zvídavé uvědomování do toho, co právě děláte. Napojte se na své smysly (např. co vidím, zvuky, doteky, pachy).}
}

	%\clearpage
	\subsection{Mé osobní hodnoty \normalPencilLeftDown}
		Co je pro vás, úplně uvnitř a niterně, důležité? Na čem byste chtěli, aby váš život stál?
		\begin{center}
		\begin{tikzpicture}
			% https://tex.stackexchange.com/a/145381/6758
			\draw (0:0) node[circle,draw,inner sep=.05\linewidth](self){já};
			\foreach \a in {1,2,...,6}{
				\draw (13+\a*360/6: .3\linewidth) node[circle,draw,inner sep=.08\linewidth](circ){ };
				\draw (self)--(circ);
			}
		\end{tikzpicture}
		\end{center}

		Osobní hodnoty jsou principy a přesvědčení o tom, jak chceme žít a jací chceme být. Hodnoty jsou směry, ve kterých se pohybujeme. Například, pokud chcete být milujícím, laskavým a podpůrným partnerem, to je hodnota — pokračující proces.

		Použijte diagram jako pomůcku, jak se na své hodnoty podívat. Do každého políčka vepište nějakou svou hodnotu. Nemusíte využít všechna políčka, a můžete si přikreslit další dle potřeby. \emph{Jako pomůcku pro reflexi o svých hodnotách můžete využít následující otázky níže.}

		Uvedené oblasti života představují pro některé lidi hodnoty. Ne všichni máme hodnoty shodné, nejde zde o ověření toho, zda máte hodnoty „správné“. Některé z těchto oblastí pro vás možná důležité nejsou; můžete je klidně přeskočit.
		% style=unboxed: allow linebreak in item title
		%\def\valItem#1{\item[#1]}
		\bgroup
		\parskip0pt
		\def\valItem#1{\par\vskip0pt plus.5em\textbf{#1} }
		\setstretch{.95}
		%\begin{description*}[leftmargin=0pt]
			\valItem{Rodina.}  Jaká/jaký chcete být jako sestra/bratr, dcera/syn, teta/\allowbreak{}strýc, člen rodiny? Které osobní vlastnosti byste rád(a) vnesl(a) do těchto vztahů? Jaké vztahy bych chtěl(a) vybudovat? Jak byste se choval(a) k ostatním, pokud byste byl(a) své „ideální já“?
			\valItem{Manželství, partnerství, intimní vztahy.}  Jakou partnerkou / jakým partnerem byste chtěl(a) být v intimním vztahu? Které osobní kvality byste rád(a) rozvíjel(a)? Jaký je vztah, který byste chtěl(a) vybudovat? Jak byste se choval(a) ke své partnerce / svému partnerovi, pokud byste byl(a) své „ideální já“?
			\valItem{Rodičovství.}  Jakým rodičem byste chtěl(a) být? Které vlastnosti byste chtěl(a) mít jako rodič? Jaký vztah byste rád(a) vybudovala se svými dětmi? Jak byste se chovala, pokud byste byla své „ideální já“ v roli rodiče?
			\valItem{Přátelství.} Které kvality byste rád(a) vnášel(a) do svých přátelských vztahů? Pokud byste byl(a) nejlepší možná přítel(kyně), jak byste se choval(a) ke svým přítelům? Jaká přátelství byste rád(a) budovala?
			\valItem{Kariéra, práce.} Čeho si ceníte na své práci? Co by ji udělalo více smysluplnou? Jaká/jaký chcete být jako zaměstnanec?  Pokud byste žil(a) podle své ideální představy, které osobní kvality byste do práce přinášel(a)? Jaké typy vztahů na pracovišti byste rád(a) budovala?
			\valItem{Vzdělávání, osobní růst a rozvoj.}  Čeho si na učení, vzdělávání, tréninku či osobním rozvoji ceníte? Které nové dovednosti byste se rád(a) naučil(a)? O čem byste rád(a) získala vědomosti? Jaké další vzdělávání či učení vás přitahuje? Jakou studentkou / jakým studentem byste chtěl(a) být? Které osobní kvality byste v této oblasti chtěl(a) využít?
			\valItem{Rekreace, zábava, oddych.}  Které zájmy, sporty či oddechové aktivity máte rád(a)? Jak byste rád(a) relaxoval(a) a odpočíval(a)? Jakému typu zábavy byste se rád(a) oddával(a)? Kterým aktivitám byste se rád(a) věnoval(a)?
			\valItem{Spiritualita / duchovno.}  Duchovno znamená pro různé lidi různé věci. Může to být spojení s přírodou, může to být účast na náboženské skupině. Co je pro vás v této oblasti důležité?
			\valItem{Občanská společnost, životní prostředí, život obce.}  Jak byste rád(a) přispěl(a) své obci či prostředí, např. dobrovolnickou činností, recyklací, podporou spolku / charity / politické strany? Jaké prostředí byste si rád(a) vytvořil(a) doma, v práci, v obci? V jakém prostředí byste rád(a) trávil(a) víc času?
			\valItem{Zdraví.}  Jaké máte hodnoty v oblasti zachování fyzické pohody? Jak se chcete starat o své zdraví s ohledem na spánek, stravování, cvičení, kouření, pití alkohou, …? Proč je to pro vás důležité?
		%\end{description*}
		\egroup

	%\clearpage
	\vskip-2mm
	\subsection{\textsc{Brzda} s hodnotami}
		\vskip-2mm
		\emph{\textup{\textsc{Brzda}} je cvičení všímavosti „za pochodu“, které můžete použít kdekoliv a kdykoliv, protože je krátké, jednoduché a přizpůsobivé.} \textup{\textsc{Brzd}}u můžete použít jako nástroj, který vám pomůže se vědomě rozhodovat a následovat své hodnoty a cíle.
		% MBSR-sesit: STOP
		\begin{itemize*}[leftmargin=10mm]
		\itemStop{B}{Brzdi.} Úplně nejdřív šlápněte na brzdu. Je to první krok, kterým vystoupíte z autopilota.
		\itemStop{R}{Roz-vhled.} Rozhlédněte se ve své svou momentální zkušenosti, okolo sebe i v sobě (tělo, pocity, myšlenky). Zkuste se dívat s určitou zvídavostí a bez odsuzování.
		\itemStop{Z}{Zakotvi se.} Dejte si několik pomalých nádechů a výdechů a zakotvěte přitom svou pozornost v tělesných počitcích, které dýchání doprovázejí. % FIXME: počitky? vejmy? pošitky jinak nikde nemám
		\itemStop{D\kern-.1em}{Doširoka se otevři.} Rozšiřte pozornost od dechu na tělo i na celou situaci. \emph{Otevřete se ještě šířeji a zvažte i své hodnoty.}
		\itemStop{A}{Akce.} Jednete v nastalé situaci s vnitřní orientací. \emph{Jak vám vaše hodnoty v moudré reakci pomáhají?}
		\end{itemize*}
	\vfill

	% FIXME: těžké rozhodování? zásek? Váhání?
	\subsection{\textsc{Brzda} na váhání: úzkost}
	\soberSpace{
		title1=Situace,
		title2={Mám úzkost před pracovním pohovorem.\\ Nejsem si jistý, zda na něj mám jít.},
		autopilot={
			thought/{Určitě se ztrapním a řeknu nějakou blbost.},
			thought/{Proč tam vůbec chodit, stejně mě nevezmou.},
			action/{Úzkost je silnější, k tomu se přidává stud.},
			action/{Na pohovor nejdu, zůstávám doma a ležím v posteli.}
		},
		brzda B={Zastavte se a vyskočte z autopilota.},
		brzda R={{srdce rychle bije;},{úzkost, obavy;},{myšlenky jak se vyhnout pohovoru.}},
		brzda Z={Několikrát se pomalu a pozorně nadýchěte a vydýchněte. Pozornost zaměřte na dech.},
		brzda D={Rozšiřte znovu svou pozornost na sebe i celou situaci. Pak pozornost rozšiřte ještě víc a zvažte i své osobní hodnoty.},
		brzda A={{Řeknu si: \toSelf{Získat práci je teď pro mě důležité. Udělám to nejlepší, co svedu.}},{Jdu na pohovor.}}
	}
	\clearpage
	\subsection{\textsc{Brzda} na váhání: nuda}
	\soberSpace{
		title1=Situace,
		title2={Nedělní nuda. Nevím, co sám se sebou.},
		autopilot={
			thought/{Co bych tak mohl dělat? Možná se budu jen tak flákat.},
			thought/{Byl bych uvolněnější, kdybych si něco malého dal.},
			action/{Cítím bažení po droze.},
			action/{Hledám doma, jestli ještě někde nemám alkohol či drogy.}
		},
		brzda B={Zastavte se a vyskočte z autopilota.},
		brzda R={{napětí v ramenou;},{vnitřní nepokoj;},{myšlenky \toSelf{co bych tak mohl dělat?}}},
		brzda Z={Několikrát se pomalu a pozorně nadýchěte a vydýchněte. Pozornost zaměřte na dech.},
		brzda D={Rozšiřte znovu svou pozornost na sebe i celou situaci. Pak pozornost rozšiřte ještě víc a zvažte i své osobní hodnoty.},
		brzda A={{Řeknu si: \toSelf{Vlastně mám rád svou dceru; rád bych s ní víc hodnotně trávil čas.}},{Volám dceři, naplánuju, že se sejdeme na oběd.}}
	}
	% no own SOBER handouts here?

	\pagePracticeLog

\mbrpSession{
	title={Prozkoumat své potřeby},
	summary={
		\item K alkoholu/drogám se můžeme obrátit, když se snažíme naplnit své potřeby v daném okamžiku. Mezi tyto potřeby patří i ty prospěšné a zdravé, které všichni jako liské bytosti máme: např. potřeba pocitu bezpečí, úlevy, vlivu na události, napojení na ostatní, pocitu štěstí a živosti.
		\item Návykové látky nám nedají to, co slibují. V dlouhodobé perspektivě užívání alkoholu/drog naše potřeby nenaplňuje. Je samozřejmě zcela pochopitelné, že se k nim obrátíme. Mít nutkání užít alkohol/drogu či návykové látky vyhledávat není nijak „špatně“. V těchto chvílích se snažíme pečovat o sebe a naplnit své potřeby, jako všichni ostatní lidé.
		\item Všímavost nám pomáhá dívat se „pod“ prvotní nutkání k užití alkoholu/drogy a prozkoumat své hlubší potřeby, které v tu chvíli možná máme. Když nutkání přijde, můžeme se zabrzdit a zeptat se sami sebe: \toSelf{Co v tuto chvíli opravdu potřebuji?} Rozeznání toho, co v tu chvíli opravu potřebujeme, nám pomáhá rozhodovat se moudře — místo automatického obrácení se k alkoholu/drogám.
	},
	formal={\item napojení na sebe;\item meditace zkoumání potřeb.},
	informal moments={Náhodně během dne se zastavujte a napojujte se na sebe.},
	informal challenges={Používejte \textsc{Brzd}u, když se objeví obtíže či nutkání. Zvažte své potřeby. Zeptejte se sami sebe: \toSelf{Co v tuto chvíli opravdu potřebuji?}},
	informal activities={Cvičte všímavost k přírodě, když jste venku. Vneste zvídavé uvědomování do různých prožitků v přírodě, když vnímáte např. oblohu, oblaka, vítr, rostliny, stromy či zvířata.}
}
	\pageBasicNeeds
	\pagePracticeLog

	
\endinput


\bgroup
	\cleardoublepage
	\setcounter{section}{-1}
	\titleformat{\section}[frame]{}{}{1em}{\Large\scshape\bfseries}
	%\titleformat{\section}{}{}{0em}{\Large\scshape\bfseries}
	\section{Startovací balíček}
\egroup	

\emph{TODO: Startovací balíček existuje primárně jako webová stárnka, aktuálně \url{http://lessstress.cz/mbrp/brzda/start}. Její obsah se od původního startovacího balíčku zde již poněkud odchyluje. Pokud bude i PDF verze, je třeba ho směrem sem synchronizovat.}

\subsection{O všímavosti a MBRP}
	Všímavost je: \textbf{být vědomě s právě probíhajícím vlastním prožíváním,} (například jak se cítíme a co vnímáme okolo sebe) \textbf{s otevřeností a bez hodnocení}.

	Prevence relapsu všímavostí:
	\begin{itemize}
		\item MBRP (mindfulness-based relapse prevention) je program vytvořený k prevenci relapsu u lidí, kteří se zotavují z užívání návykových látek.
		\item Obsahem MBRP je učení se trénování dovednosti být všímavý pomocí různých vedených cvičení.
		\item MBRP vám může pomoci:
		\begin{itemize*}
			\item Lépe si uvědomovat spouštěče relapsu.
			\item Zvládat spouštěče (nejen na ně reagovat).
			\item Pracovat s obtížnými pocity zdravým způsobem.
			\item Být k sobě laskavější.
			\item Kultivovat životní styl, který podporuje dlouhodobé uzdravení.
		\end{itemize*}
	\end{itemize}
	Proč všímavost? Výzkum ukazuje, že učení se a cvičení všímavosti:
	\begin{itemize}
		\item[\mbrpIcon{trend-down}] Snižuje riziko relapsu do užívání alkoholu/drog.
		\item[\mbrpIcon{trend-down}] Snižuje nutkání/bažení po užívání alkoholu/drog.
		\item[\mbrpIcon{trend-up}] Zlepšuje emoční spokojenost.
	\end{itemize}
	\subsection{Výzkum}
	Zde jsou výsledky výzkumu programu MBRP:

	\begin{enumerate}
		\item Bowen, S., Witkiewitz, K., Clifasefi, S. L., Grow, J., Chawla, N., Hsu, S. H., \& Larimer, M. E. (2014). \emph{Relative efficacy of mindfulness-based relapse prevention, standard relapse prevention, and treatment as usual for substance use disorders: a randomized clinical trial.} JAMA Psychiatry, 71(5), 547-556.
		\item Li, W., Howard, M. O., Garland, E. L., McGovern, P., \& Lazar, M. (2017). \emph{Mindfulness treatment for substance misuse: A systematic review and meta-analysis.} Journal of Substance Abuse Treatment, 75, 62-96.
	\end{enumerate}

	\subsection{Plán pravidelného cvičení \normalPencilLeftDown}
	\subsubsection{Příklady}
		\begin{description}
			\item[Honza:] \emph{„Přibližně 6 dní v týdnu poslouchám 5–10 minut nahraných cvičení. Cvičím před snídaní. V sedě na židli. Cvičit každý den ve stejný čas mi pomáhá na cvičení nezapomenout. Otevřu si lessstress.cz na telefonu a tam si nahrávky pouštím. Cvičení všímavosti je součást mého života, protože mi pomáhá více si uvědomovat, co se děje, a díky tomu se mohu lépe udržet na cestě zotavení.“}
			\item[Alena:] \emph{„Posouchám 10- či 15-minutová cvičení všímavosti asi tak 5 dní v týdnu. Cvičím večer před spaním, když už jsem po sprše v pyžamu. Sedím na posteli a poštáři si vypodložím záda. Nahrávky si pouštím na MP3 přehrávači, který jsem si půjčila od svého terapeuta. Mám slouchátka, aby mě méně rušil hluk zvenku. Když se držím pravidelného cvičení, cítím se více ukotvená a lépe zvládám stres, který se v mém životě objeví.“}
		\end{description}
	\subsection{Jak si pustit nahrávky cvičení}
		\textbf{TODO: add instructions, as if given as the starting pack to the participant.}


	\subsection{Nahrávky vedených meditací}
		TODO: make list complete
		\begin{description*}
			\item[Napojení na sebe:] všímání si toho, co se děje v našem těle a mysli, následované zaměřením pozornosti na dech. \par Napojení na sebe je základní cvičení, které zařazujeme na začátek každého setkání. Je vhodné pro opakované poslouchání a cvičení. Samozřejmě ho střídejte i s jinými cvičeními, například těmi uvedenými níže.
			\item[Body scan:] procházení těla pozorností, přinášení zvídavosti k tělesným vjemům.
			\item[\textsc{Brzda}:] pět kroků zkratky: Brzdi, Roz-vhled, Zakotvi se, Doširoka se otevři, Akce.
			\item[Jízda na vlně nutkání:] všímavě se držet na vlně nutkání či bažení, jak v čase sama zesiluje a zase slábne.
			\item[Meditace s dechem:] TODO
			\item[Meditace o hoře:] použití vizualizace k rozvinutí pocitu síly a stability.
			\item[Meditace laskavosti:] posílání myšlenek a přání laskavosti a soucitu sobě a ostatním.
			\item[Všímavý pohyb:] pěstování zvídavé pozornosti k tělesným vjemům při jemných pohybech a protahování.
			\item[Všímavá chůze:] přinášení zvídavé pozornosti k tělesným vjemům při chůzi.
			%\item[Meditace o myšlenkách:] všímavé pozorování myšlenek, jak procházejí vědomím.
		\end{description*}


	\clearpage
	\subsection{Vlastní plán \normalPencilLeftDown}
		\paragraph{Počet dní v týdnu, kdy budu cvičit s nahrávkou:} \Blank{1}\par Doporučujeme si pustit nahrané cvičení denně či obden (4–6 dní v týdnů). Pravidelné cvičení udrží váš „sval všímavosti“ v kondici.
		\paragraph{Denní doba, kdy budu obvykle cvičit:} \Blank{2} \par Zde je pár příkladů denní doby, která se používá:
			\begin{enumerate*}
				\item Ráno: před snídaní, po sprše, při přípravě na den.
				\item Večer: po převlečení do pyžama, během večer ukládací rutiny.
				\item Přes den: během přestávky na oběd.
			\end{enumerate*}
		\paragraph{Jak si budu pouštět nahrávky? [?]} \Blank{2}
		\paragraph{Kde budu cvičit?} \Blank{2}\par Zde je pár možností: ložnice, v autobuse či vlaku (na sluchátka), v autě (zaparkovaném), v kanceláři, venku.
		\paragraph{V jaké pozici?} \Blank{2} \par Doporučujeme sedět vzpřímeně, neopírat se. Podle potřeby použijte polštářky jako oporu pro záda.
			\begin{enumerate*}
				\item V sedě na židli, chodidla na zemi.
				\item V sedě na pohovce, nohy zkřížené či na zemi.
				\item V sedě na posteli, zkřížené nohy.
				\item V sedě na zemi (je možné se opřít zády o zeď).
			\end{enumerate*}
			Lze cvičit i v leže. V této pozici je však snažší upadnout do ospalosti. Pokud byste se v leže cítili velmi ospalí, zkuste jednu z uvedených vzpřímených pozic.
	\clearpage
	\subsection{Všímavost za pochodu, během všedního dne}
		\begin{description}
			\item[\mbrpIcon{pause} Všímavé zastavení.] Náhodně během dne se \emph{zastavte} a 
				\begin{itemize*}
					\item zpomalte se několika vědomými dechy;
					\item dejte si 1 minutu, abyste se na sebe napojili (Jak se právě teď cítím?).
				\end{itemize*}
			\item[\mbrpIcon{head-brain} Všímavé zvládání obtíží.]
				Když se objeví náročná situace, např. hádka, obtížné pocity, nutkání k užití…:
				\begin{itemize*}
					\item projděte situací všímavě, méně reaktivně;
					\item použijte \textsc{Brzd}u.
				\end{itemize*}
			\item[\mbrpIcon{apple} Všímavé činnosti.]
				Když se věnujete běžným činnostem, jako je např. jedení, domácí práce, sprchování se, čištění zubů, chůze někam…:
					\begin{itemize*}
						\item vneste do přítomného okamžiku zvídavou pozornost;
						\item napojte se na 5 smyslů (zrak, sluch, chuť, čich, hmat).
					\end{itemize*}
		\end{description}
		\vfill

	\pageBRZDA
	\pagePracticeLog

\subsection{Zvládání obvyklých obtíží při cvičení}

	{\setstretch{1.01}
	Je zcela v pořádku, normální a běžné zakoušet při cvičení obtíže, které jsou níže rozebrané — i nějaké jiné. Není na nich nic „špatného“ či „vadného“. Nejsou poruchou procesu, naopak, jsou jeho nedílnou součástí. \emph{Můžete} jimi projít a ve cvičení pokračovat. Zde jsou tipy, jak se skrz různé obtíže propracovat, pokud se objeví.
	\begin{description*}
		\item[„Moje mysl se neustále toulá.“] Je naprosto normální a běžné zakoušet toulavou mysl a mít během cvičení všímavosti spoustu myšlenek. Je to jedna z věcí, kterou mysl dělává: toulá se. Nemusíte se snažit myšlenky zastavit či je potlačit. Nakolik to jde, jen si uvědomte probíhající zkušenost, s určitou zvědavostí a bez odsuzování (\toSelf{Hele, podívej se, pozornost už zase odešla}). Některé myšlenky mohou být až znepokojivě „zrychlené“ či „neodbytné“. V těchto chvílích zkuste prozkoumat, jaké je se i k této zkušenosti postavit s přijetím a laskavostí (\toSelf{Jaké by to asi bylo, kdybych ty myšlenky jen zaznamenala a nechala je, místo zápasení s nimi a kritizování se za ně?}).
		\item[„Cítím se nepokojně a nedokážu v klidu sedět.“] Je běžné a zcela v pořádku se při cvičení cítit nepokojně. Může jít o tělesnou neposednost, mentální neklid, či oboje současně. Zkuste, co se stane, když i ten neklid zaznamenáte se zvídavostí (\toSelf{Hmm, jak to vlastně prožívám?}), namísto odsuzování sebe či snahy nepokoj zahnat. Můžete do svého cvičení, vedle těch, která jsou vsedu, přidat všímavou chůzi a všímavý pohyb.
		\item[„Jsem ospalá a při cvičení usínám.“] Usnout při cvičení je v pořádku. Netřeba se toho bát! Ospalost se objevuje víc při poloze vleže. Zkuste místo toho cvičit ve vzpřímeném sedu, případně mít pootevřené či otevřené oči (s pohledem uvolněně sklopeným šikmo dolů před sebe).
		\item[„Nezvládám to“ či „Nedělám to dobře“.] Myšlenky tohoto typu jsou zcela normální — začináme pochybovat o sobě a své schopnosti všímavost cvičit. Zkuste zlehka tyto myšlenky zaznamenat, se zvídavostí a laskavostí (\toSelf{Helemese, už zase přišlo tohle pochybování.}). Připomínejte si, že není nic takového jako cvičit „správně“ či „špatně“. Je jen provádění cvičení a zjistit, co přijde. Pokud se vaše mysl hodně toulá či je pro vás těžké zaměřit pozornost, je to zcela v pořádku a normální.
		\item[„Necítím se o nic lépe. Co je to se mnou?“] Při cvičení je běžné cítit touhu, abychom se cítili lépe, či se soudit za to, že se cítíme tak, jak se cítíme. Zkuste se na chviličku zastavit a vzít na vědomí tuto touhu se cítit jinak (\toSelf{Páni, teď vidím, jak sama sebe dostávám pod tlak, když se chci cítit jinak, než se cítím.}). Jaké by bylo dovolit si cítit se v tuto chvíli právě tak, jak se cítíte? Cvičení všímavosti zahrnuje kultivaci přijímání k naší prožívané zkušenosti — nejde o vytváření tlaku na sebe, abychom se cítili jinak či lépe.
		\item[„Pořád na cvičení zapomínám.“] Je úplně normální a v pořádku na cvičení zapomenout či mít období, kdy jsme zaneprázdněni či nás něco od cvičení odtáhne. To se děje nám všem. I pokud jste zmeškali něklik dní či týdnů cvičení, kdykoliv se můžete ke cvičení vrátit a pokračovat. Důležité je mít vnitřní závazek k pravidelnému cvičení v dlouhém časovém horizontu a ke cvičení se vracet znova a znova. Připomeňte si, že na cvičení všímavosti nemůsíte ze svého dne ukrojit obrovské porce času. I kdyby to bylo 5 či 10 minut denně, dlouhodobě to přinese užitek. Dělat malé cvičení každý den je jako každý den mozek „dobíjet“, aby zůstal silný a zdravý. Můžete si také nastavit alarm na telefonu, aby vám cvičení připomenul.
	\end{description*}
	}
\subsection{Posílení motivace ke cvičení \normalPencilLeftDown}
	Jaké jsou vaše osobní motivate ke cvičení všímavosti? Zaškrtněte některé z uvedených možností, případně dopište své vlastní osobní motivy.

	%\begin{tikzpicture}[every node/.style={draw,text centered,text depth=4\baselineskip,rounded corners}]
	%	\matrix[matrix of nodes,text width=.28\linewidth,draw=none,column sep=.03\linewidth,row sep=1em]{
	%		Zotavení ze závislosti je pro mě důležité. &
	%		Péče o sebe je důležitou součástí mého života. &
	%		Chci posílit svou schopnost zvládání stresu v životě. \\
	%		Rád zkouším nové věci. &
	%		Je pro mě důležité, že všímavost je podepřená výzkumem. &
	%		Uvědomuju si, že ostatím všímavost v zotavení pomohla. \\
	%		Chci se naučit účinné strategie zvládání životních situací. &
	%		Svého mentálního zdraví si cením minimálně tak jako tělesného. &
	%		Chci žít zdravý a vyvážený život. \\
	%		… & … & … \\
	%	};
	%\end{tikzpicture}

	\begin{itemize}
		\itemSq Zotavení ze závislosti je pro mě důležité.
		\itemSq Péče o sebe je důležitou součástí mého života.
		\itemSq Chci posílit svou schopnost zvládání stresu v životě.
		\itemSq Rád zkouším nové věci.
		\itemSq Je pro mě důležité, že všímavost je podepřená výzkumem.
		\itemSq Uvědomuju si, že ostatím všímavost v zotavení pomohla.
		\itemSq Chci se naučit účinné strategie zvládání životních situací.
		\itemSq Svého mentálního zdraví si cením minimálně tak jako tělesného.
		\itemSq Chci žít zdravý a vyvážený život.
		\itemSq \Blank{1}
		\itemSq \Blank{1}
		\itemSq \Blank{1}
	\end{itemize}

	\clearpage
	\subsection{Proč u toho vydržet?}

		\mbrpIcon{head-brain} Myslete na svůj MOZEK! 

		\textbf{Naše mozky doslova rostou a vytvářejí nová spojení po celý život}, včetně dospělosti. Nenarodíme se jednoduše s mozkem, který nám už pak na celý život zůstane. Náš mozek se neustále proměňuje podle zkušeností, které prožíváme, a informací a dovedností, které se učíme a cvičíme. Kdykoliv se učíme či cvičíme něco nového, např. nějakou dovednost, vytvářejí se v mozku nové spoje. S dalším a dalším cvičením té dovednosti se tyto spoje dále zesilují. \emph{Neuroplasticita} (dosl. nervová tvárnost) je odborné slovo, kterým se tyto změny v mozku popisují.

		\mbrpIcon{lifting-weight} \mbrpIcon{biceps-flex}
		\textbf{Váš mozek je jako sval, který je možné posílit úsilím a cvičením.} Podobně jako fyzická cvičení posilují tělo, cvičení mozku mentálními cvičeními — jako je např. všímavost — posiluje váš mozek. Výzkum ukazuje, že pravidelné cvičení všímavosti mozek pozorovatelně mění a posiluje oblasti mozku, které se podílejí na zvládání stresu a emocí. Tyto studie též ukázaly, že hladina stresu se pravidelným cvičením všíavosti snižuje.

		
		\mbrpIcon{trend-up}
		\textbf{Zapamatujte si: máte schopnost změnit svůj mozek. Všímavost je vědou podepřený nástroj, který můžete používat ke změně mozku a zvýšení schopnosti zvládat náročné situace.}

