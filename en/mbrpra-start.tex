\bgroup
	\cleardoublepage
	\setcounter{section}{-1}
	\titleformat{\section}[frame]{}{}{1em}{\Large\scshape\bfseries}
	%\titleformat{\section}{}{}{0em}{\Large\scshape\bfseries}
	\section{Getting Started}
\egroup	



\subsection{About Mindfulness-Based Relapse Prevention}

	Mindfulness-Based Relapse Prevention (MBRP) is a program designed to prevent relapse, in prevention of or recovery from addiction — in the wide sense (repeated behavior leading to significant harm and distress), including:
	\begin{itemize*}
		\item substance addictions (alcohol, drugs, tobacco, medicaments);
		\item addictive behaviors:
			\begin{itemize*}
				\item external: food, work, sport, sex, pornography, gambling, self-harm, shopping, …;
				\item  internal: frequent self-criticism or devaluation of others, aggressiveness, anxiety, submissiveness, …
			\end{itemize*}
	\end{itemize*}

	\textbf{Mindfulness} is being aware of our present moment experience (such as how we feel or what is going on around us) in an open and nonjudgmental way.

	MBRP is a structured training of the skill of mindfulness using various guided exercises. The ability to stay oriented in one’s experience as it unfolds reduces the likelihood of relapse — falling back into the addictive behavior.

	\clearpage
	MBRP can help you:
	\begin{itemize*}
		\item Be more aware of relapse triggers.
		\item Respond/cope with triggers (not just react).
		\item Work with difficult emotions in healthy ways.
		\item Be kinder and more compassionate towards yourself.
		\item Develop a lifestyle that promotes long-term recovery.
	\end{itemize*}

	\subsubsection*{Research of mindfulness and MBRP}
		Research shows that learning and practicing mindfulness…
			\begin{itemize*}
				\item[\DOWNarrow] Reduces the odds of relapsing to alcohol or drug use.
				\item[\DOWNarrow] Reduces urges/cravings to use alcohol or drugs.
				\item[\UParrow] Improves emotional well-being.
			\end{itemize*}
		Research results are described in articles referenced at XXX.

	\subsubsection*{Structure}
	The MBRP programme has three pillars, main components. They are all recommended, but also voluntary.
	\begin{description*}
    	\item[Formal mindfulness practices:] regular (e.g. daily) practice with guided meditation recordings, in dedicated time.
		\item[Informal practices:] bringing mindfulness into activities and situations throughout your daily life.
    	\item[Support group meetings:] sharing the experience, support from the group, regularity and structure for the practice. Groups are facilitated by a trained professional.
	\end{description*}
		
		%Here are a few select studies:
		%\begin{enumerate}
		%	\item Bowen, S., Witkiewitz, K., Clifasefi, S. L., Grow, J., Chawla, N., Hsu, S. H., \& Larimer, M. E. (2014). \emph{Relative efficacy of mindfulness-based relapse prevention, standard relapse prevention, and treatment as usual for substance use disorders: a randomized clinical trial.} JAMA Psychiatry, 71(5), 547-556.
		%	\item Li, W., Howard, M. O., Garland, E. L., McGovern, P., \& Lazar, M. (2017). \emph{Mindfulness treatment for substance misuse: A systematic review and meta-analysis.} Journal of Substance Abuse Treatment, 75, 62-96.
		%\end{enumerate}

\clearpage
\subsection{Formal Practices}
	There are 12 guided meditations altogether, listed below. Each session recommends practices for the next week.
	\begin{itemize*}
		\item All recorded meditations are available at \href{https://PracticeMBRP.com}{PracticeMBRP.com}, you can play them on a computer or phone.
		\item You can also download and use the recordings in other suitable ways (offline from the phone, MP3 player, mindfulness app)
	\end{itemize*}
	It is beneficial to establish regularity of the practice, a routine. Example practice routines:
	\begin{description*}
		\item[John:] \emph{I listen to a 5- or 10-minute audio-guided mindfulness exercise about 6 days a week. I practice before breakfast in the morning. I sit in a chair. Practicing at the same time every day helps me remember to practice. I go to the website PracticeMBRP.com on my smartphone and play the recordings from there. I make mindfulness practice a part of my life because it helps me be more aware of what is going on and helps me stay on track with my recovery.}
		\item[Emily:] \emph{I listen to a 10 or 15-minute audio-guided mindfulness exercise about 5 days a week. I practice at night after I shower and put on my pajamas. I sit in my bed and use pillows to support my back. I play the recordings from a MP3 player I borrowed from my therapist. I also use headphones to block out other noise. When I stick to a regular practice routine, I feel more grounded and I am better able to deal with stress that comes up in my life.}
	\end{description*}

	\clearpage
	\subsubsection*{Audio-guided exercises}
	\begin{description*}
		\item[Mindful Check-in:] Check-in with yourself. Notice what’s going on in your body and mind. Then, focus your attention on your breathing. \par The Mindful Check-in is a core exercise in this program that we practice at the beginning of every session. This a good practice to listen to on a regular basis.Of course, mix it up too and listen to other exercises, such as the one’s below.
		\item[Body Scan:] “Scan” your body and bring a curious awareness to sensations you notice.
		\item[\textsc{Sober} Space:] Go through the steps of the SOBER: Stop, Observe, Breathe, Expand, Respond.
		\item[Urge Surfing:] Practice mindfully “surfing” or riding out the experience of an urge or craving.
		\item[Breath Meditation:] Focus your attention on the breath. When your attention wanders, gently bring your focus back to the breath again and again.
		\item[Mountain Meditation:] Use visualization to develop a sense of strength and stability.
		\item[Kindness Meditation:] Send thoughts and feelings of kindness and compassion to yourself and others.
		\item[Mindful Movement:] Bring a curious awareness to sensations in your body as you engage in gentle movements and stretches.
		\item[Mindful Walking:] Bring a curious awareness to sensations in your body as you walk.
		\item[Meditation on Thoughts:] Practice mindfully observing thoughts that pass through your awareness.
	\end{description*}

\clearpage
\subsection{On-The-Go (informal) Practices}
	You can use on-the-go practices at any moment of your daily life.
	\begin{description*}
		\item[Mindful Moments:] At random times during your day, \emph{pause} and…
			\begin{itemize*}
				\item Take a few mindful breaths to slow down 
				\item Take 1 minute to mindfully check-in with yourself (\toSelf{How am I feeling right now?})
			\end{itemize*}
		\item[Mindful Coping:] When challenging situations come up, such as arguments, difficult emotions, craving to use…
			\begin{itemize*}
				\item Use mindfulness to cope
				\item Use the \textsc{Sober} Space: is an \emph{on-the-go} mindfulness practice that you can do anywhere, anytime because it is brief, simple, and flexible. It is described on page \pageref{sober-space} in detail.
			\end{itemize*}
		\item[Mindful Activities:] When you are engaging in daily activities, such as eating, chores, showering, brushing teeth, walking…
			\begin{itemize*}
				\item Bring a curious attention to the present moment
				\item Connect with your 5 senses (sight, sound, touch, smell, taste)
			\end{itemize*}
	\end{description*}

\clearpage
\subsection{Common challenges}
\bgroup\setstretch{1.0}
	It is completely okay, normal, and common to experience the challenges listed below. These challenges are \emph{not} “bad” or “wrong” in any way. These challenges are simply part of the process. You \emph{can} work through these challenges and stick with your practice. Here are some tips for working through the different types of challenges that come up.
	\begin{description*}
		\item[“My mind won’t stop wandering.”] It is totally normal and common to experience a wandering mind and have lots of thoughts come up during mindfulness practice. That is just what the mind does; it wanders. You don’t need to stop thoughts or push them away. Do your best to just notice what your experience is with a sense of curiosity and nonjudgment (\toSelf{Oh look at that, there my mind goes again}). Sometimes thoughts can be “racing” or “intrusive” to the point it is very distressing. In these moments, see what it’s like to bring a sense of acceptance and kindness to your experience (\toSelf{Hmm, what would it be like to just notice these thoughts and let them be, instead of fighting them or criticizing myself?}).
		\item[“I feel restless and can’t sit still.”] It is common and totally okay to feel restless while practicing. This can be a physical restlessness and/or a mental restlessness. See what it’s like to notice the restlessness with a sense of curiosity (\toSelf{Hmm, what does this actually feel like?}), rather than judging yourself or trying to force the restlessness to go away. You can also mix in mindful walking and mindful movement into your practice, instead of only practices where you sit still.
		\item[“I feel sleepy or fall asleep when I practice.”] It’s okay if you fall asleep. No worries! Getting sleepy is more likely to happen when lying down. Try practicing sitting upright instead, or practicing with your eyes open or half open (softly gazing downwards in front of you).
		\item[“I can’t do this” or “I’m not doing it right.”] It is totally normal to have thoughts like this where we start to doubt ourselves and our ability to practice mindfulness. Try to gently acknowledge these thoughts with a sense of curiosity and kindness. (\toSelf{Oh, look at that, there’s that doubt coming up again}). Keep in mind that there is no such thing as doing a practice “right” or “wrong.” There is simply doing a practice and seeing what comes up for you. If your mind wanders a lot or you have difficulty focusing, that is totally okay and normal.
		\item[“I don’t feel any better. What’s wrong with me?”] When we practice, it is common to feel a desire to feel better or to judge ourselves for how we feel. See if you can pause and acknowledge that desire to feel different (\toSelf{Oh wow, look at that, I am putting pressure on myself to feel different than how I feel}). See what it’s like to give yourself permission to just feel the way you are feeling in the moment. Mindfulness practice involves bringing a sense of acceptance towards our experience, instead of pressuring ourselves to feel better or different.
		\item[“I keep forgetting to practice.”] It is completely normal and okay to forget to practice and to have times when you are busy or get sidetracked from your practice. This happens to all us. Even if you have missed several days or weeks of practice, you can always get back on track at any time and start practicing again. The key is to commit to regular practice for the long-run and to keep coming back to your practice again and again. Remember that you don’t need to take huge chunks out of your day to practice mindfulness. Just 5 or 10 minutes of practice a day, goes a long way! Doing little bits of practice each day is like giving your brain a little “recharge” each day, which keep your brain strong and healthy. You can also set an “alarm” on your smartphone to remind you to practice.
	\end{description*}
\egroup

\subsubsection*{Why stick with it?}
	Think about your \emph{brain}!
	\begin{description*}
		\item[Our brains literally continue to grow and form new connections all throughout our lives,] even as adults. We are not simply born with one brain that stay the same all of our lives. Rather, our brain is constantly changing depending on what experiences we have and what information and skills we learn about and practice. Whenever we learn or practice something new, like a skill, new connections are formed in the brain. The more and more we practice a skill, the stronger these connections become. The fancy word to describe changes in the brain is called neuroplasticity.
		\item[Your brain is like a muscle that can be strengthened through hard work and practice.] Just like exercising your body makes your body stronger, exercising your brain with mental exercises like mindfulness actually makes your brain stronger. Research shows that regular practice of mindfulness actually changes people brains and strengthens the areas of the brain involved in managing stress and emotions. These studies also show that people’s stress levels go down with regular practice of mindfulness.
	\end{description*}
	\emph{Remember, you have the ability to change your brain. Mindfulness is a science-backed tool you can use to change your brain and boost your coping power.}
