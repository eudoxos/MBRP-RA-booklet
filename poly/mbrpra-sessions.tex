
%%
%% 1
%%
\mbrpSession{
	title={\gettext{Mindfulness vs.\\ Automatic Pilot}},
	toc title={\gettext{Mindfulness vs. Automatic Pilot}},
	summary={\gettext{
		\item “Automatic pilot” is when we are not aware of what we are doing. We just do things automatically.
		\item When we have urges or cravings to use alcohol or other drugs, we often go into automatic pilot. We just \emph{react} without awareness.
		\item Mindfulness is the opposite of automatic pilot. When we are mindful, we are fully aware of what we are doing in the moment.
		\item Mindfulness helps us step out of automatic pilot mode and make more conscious choices (such as when we have an urge or craving to use)
	}},
	formal={\gettext{\item Mindful Check-In; \item \textsc{Sober} Space.}},
	informal moments={\gettext{At random times, pause and mindfully check-in with yourself.}},
	informal challenges={\gettext{Try doing the \textsc{Sober} Space when challenges come up.}},
	informal activities={\gettext{Try mindful eating. For each meal, take one mindful bite. Bring your full attention to your food and the experience of eating.}},
	practice sheets={\gettext{\item \textsc{Sober} Space for feelings 2×;}}
}
		
	\pageSOBER
	
	\clearpage
	\subsection{\gettext{\textsc{Sober} Space for feelings: pain and depression}}
	\soberSpace{
		title1=\gettext{Situation},
		title2={\gettext{Wake up in the morning with intense physical pain and feeling depressed}},
		autopilot={
			thought/{\gettext{Not this again. I can’t handle this.}},
			action/{\gettext{Just lie in bed, instead of getting up.}},
			thought/{\gettext{This pain will never go away.}},
			thought/{\gettext{I need to get high. \emph{(Feel craving to use)}}},
			action/{\gettext{Text a friend to get drugs.}}
		},
		sober O={\gettext{{Pressure and aching in back},{Sadness},{Lots of negative thoughts popping up}}},
		sober R={\gettext{{Remind yourself: \toSelf{I can handle this. I have done it before.}},{ Get out of bed and take a hot shower.}}}
	}

	\clearpage
	\subsection{\gettext{\textsc{Sober} Space for feelings: boredom and loneliness}}
	\soberSpace{
		title1={\gettext{Situation}},
		title2={\gettext{Feel bored and lonely on a Saturday night}},
		autopilot={%
			thought/{\gettext{My life is so boring.}},
			thought/{\gettext{I could use some company right now.}},
			action/{\gettext{Call a friend you previously used alcohol and drugs with.}},
			action/{\gettext{Feel craving to use.}}
		},
		sober O={\gettext{{Heaviness in chest area},{Bored, Lonely},{Thoughts about reaching out to others.}}},
		sober R={\gettext{
			{Call a sober-support friend and make plans to get dinner.}
		}}
	}


%%
%% 2
%%
\mbrpSession{
	title={\gettext{Emotions as Visitors}},
	summary={\gettext{
		\item Challenging emotions (anxiety, sadness, anger) can trigger an urge to use alcohol or drugs. We may turn to alcohol and drugs to avoid or escape our emotions. 
		\item Mindfully accepting emotions means allowing ourselves to feel our emotions in the moment, instead of trying to escape our emotions. It means acknowledging emotions (not ignoring them) with a sense of openness, curiosity, and nonjudgment.
		\item Challenging emotions are like “visitors” or “guests.” They visit us and then pass by. They are temporary. Challenging emotions can also be helpful when they visit. \emph{… treat each guest honorably. / They may be clearing you out / for some new delight […] Because each has been sent / as a guide from beyond.} (\emph{The Guest House Poem}).
	}},
	formal={\gettext{\item Mindful Check-In; \item Mindfulness of Emotions (with \emph{Guest House} Reading).}},
	informal moments={\gettext{At random times, pause and mindfully check-in. Notice any emotions you might be feeling with a sense of openness and acceptance.}},
	informal challenges={\gettext{Try doing the \textsc{Sober} Space when difficult emotions come up. Practice accepting your emotions and allowing yourself to feel your emotions.}},
	informal activities={\gettext{Try mindful showering or bathing. Bring your full attention to the activity of showering or bathing. Connect with your 5 senses.}},
	practice sheets={}
}

	\pageGuesthouse
	\pageBasicFeelings


%%
%% 3
%%
\mbrpSession{
	title={\gettext{Self-Compassion}},
	summary={\gettext{
		\item Self-compassion is about being kind, gentle, and understanding towards ourselves. Self-compassion helps us work through from setbacks and slips in the recovery process.
		\item Self-compassion means not judging ourselves when we are upset or having a hard time. We can pause and tell ourselves \toSelf{It’s okay to feel upset. I am doing the best I can.}
		\item All human beings struggle. No one is perfect. When we practice self-compassion, we can tell ourselves: \toSelf{I’m not alone feeling this way. We all go through hard times.}
	}},
	formal={\gettext{\item Mindful Check-In; \item Kindness Meditation.}},
	informal moments={\gettext{At random times, pause and mindfully check-in with yourself.}},
	informal challenges={\gettext{Try doing the \textsc{Sober} Space when challenges come up. Practice being kind and compassionate to yourself.}},
	informal activities={\gettext{Try mindful teeth brushing. Bring your full attention to the activity of brushing your teeth. Connect with your 5 senses.}},
	practice sheets={}
}


%%
%% 4
%%
\mbrpSession{
	toc title={\gettext{Responding (not reacting) to triggers}},
	title={\gettext{Responding (not reacting)\\ to triggers}},
	summary={\gettext{
		\item Triggers are persons, places, or things that bring about an urge or craving to do something, such as using substances 
		\item Triggers can “set off” a chain of sensations, thoughts, and emotions, that are part of the overall craving experience. Using the \textsc{Sober} Space helps us be more aware of both the triggers itself as well as the sensations, thoughts, and emotions that come up.  
		\item Using the \textsc{Sober} Space helps us respond to triggers with awareness, instead of reacting automatically. Using the \textsc{Sober} Space helps us make conscious choices in the moment that make it less likely we will use substances when triggered.
		\item The \textsc{Sober} Space is flexible and can be used in different ways. In some cases, you may only have time to quickly “Stop” and then walk away from a trigger (person offering you drugs). Then, when you are somewhere else, you can go through the entire \textsc{Sober} to continue coping with the situation and any craving you may still be experiencing.  
	}},
	formal={\gettext{\item Mindful Check-In; \item \textsc{Sober} Space.}},
	informal moments={\gettext{At random times, pause and mindfully check-in.}},
	informal challenges={\gettext{Practice the \textsc{Sober} Space when challenges or triggers come up.}},
	informal activities={\gettext{Try mindful hearing during your day. Pause and just take in the sounds around you. Try this when you are outdoors. Try this when you are indoor too. Another options is to try mindful hearing with music. Play a song that you enjoy and practice paying close attention to what the song sounds like.}},
	practice sheets={\gettext{\item \textsc{Sober} with trigger 2×;}}
}

	\subsection{\gettext{\textsc{Sober} Space for triggers: argument}}
	\soberSpace{
		title1={{\Large\textup{\lightning}} \gettext{Primary Trigger}},
		title2={\gettext{Argument with significant other}},
		autopilot={
			action/{\gettext{Automatically walk in direction of nearby liquor store.}},
			action/{\lightning{} \gettext{\textbf{trigger}: \\ See liquor store.}},
			action/{\gettext{Go in store, buy alcohol, and start drinking.}}
		},
		sober S={\gettext{Slow down before doing anything.}},
		sober O={\gettext{
			{Tightness in chest;},
			{Angry, Guilty.},
			{Having thought \toSelf{I need a drink}.}
		}},
		sober B={\gettext{Focus on your breathing. Notice the breath flowing in and out. Slow down even more.}},
		sober E={\gettext{Expand back to the full situation. Be aware of yourself and the environment.}},
		sober R={\gettext{
			{Go to bedroom. Listen to music.},
			{Talk over things with significant other.}
		}}
	}

	\clearpage
	\subsection{\gettext{\textsc{Sober} Space for triggers: old friend}}
	\soberSpace{
		title1={{\Large\textup{\lightning}} \gettext{Primary Trigger}},
		title2={\gettext{Across the street, you see an old friend who used to sell drugs to you.}},
		autopilot={
			action/{\gettext{Automatically walk across street towards old friend.}},
			action/{\lightning{} \gettext{\textbf{Trigger}:\\ He tells you he has drugs in his car around the corner.}},
			action/{\gettext{Go with him to car around corner.}},
			action/{\lightning{} \gettext{\textbf{Trigger}:\\ He shows you the drugs.}},
			action/{\gettext{You buy the drugs and use them.}}
		},
		sober S={\gettext{Pause and step out of automatic pilot mode.}},
		sober O={\gettext{
			{Warm feeling in belly. Mouth salivating.},
			{Desire or craving to feel high;},
			{Having thought \toSelf{I should go say hello}}
		}},
		sober B={\gettext{Take a slow, mindful breath in and out.}},
		sober E={\gettext{Expand back to the full situation. Be aware of yourself and the environment.}},
		sober R={\gettext{{Walk away in the other direction.},{Take out smartphone and listen to guided \textsc{Sober} space when you get to apartment.}}}
	}


%%
%% 5
%%
\mbrpSession{
	title={\gettext{Seeing Thoughts as Thoughts}},
	summary={\gettext{
		\item Thoughts are words or images that pass through our minds. Thoughts often “pop up” automatically. Just like our lungs breathe and our heart beats, our mind thinks. That is just what it does. 
		\item Our thoughts are not necessarily true, and we don’t always need to believe in or “buy into” our own thoughts. 
		\item Automatically believing in or acting upon our thoughts (ex. \toSelf{I can’t deal with this. Getting high would make things a lot better}) can lead to substance use.
		\item Mindfulness can help us see our thoughts as just thoughts, instead of truths or commands. Mindfulness can help us “step back,” notice thoughts as they come up, and respond to these thoughts in healthy ways that prevent substance use.
	}},
	formal={\item \gettext{Mindful Check-In.}},
	informal moments={\gettext{At random times, pause and mindfully check-in with yourself.}},
	informal challenges={\gettext{Try doing the \textsc{Sober} Space when challenges come up.}},
	informal activities={\gettext{Try mindful walking. Bring your full attention to the activity of walking. Notice the sensations in your body as you walk.}},
	practice sheets={}
}
	\clearpage
	\subsection{\gettext{The Role of Thoughts in the Relapse Cycle}}
		\begin{adjustbox}{width=\linewidth}
		\begin{tikzpicture}[
			every node/.style={
				text centered,
				text width=.35\linewidth,
				draw,
				rounded corners,
				% scale=.85
			},
			node distance=2em and 1em,
			]
			\node(top){\textbf{\gettext{Trigger:}}\\\gettext{Physical pain and stress from a worsening health condition}};
			\node(initial)[thought,below=of top,callout relative pointer={(0,1)}]{\emph{\gettext{Initial Thought:}} \gettext{„I can’t deal with this. Getting high would make things a lot better.“}};
			\draw[->,color=black!40,line width=1mm] (top)--(initial);

			\node(autopilot)[autopilot,draw=none,below left=of top]{\textbf{\textsc{\gettext{Automatic pilot}}:}\\\gettext{Believe thoughts\\Act on thoughts}};
			\node(auto1)[autopilot,below left=of initial]{\gettext{Believe thoughts. Feel upset and have urge to use. Go get some alcohol or drugs.}};
			\node(auto2)[autopilot,action,below=of auto1]{\gettext{Go back home and take one pill/drink/hit (\toSelf{Just one})}};
			\node(auto3)[autopilot,thought,below=of auto2,callout relative pointer={(0,1)}]{\emph{\gettext{Thought:}}\\ \gettext{“\textbf{Ah, Screw it.}”}  \\ \gettext{“I’m a failure. I might as well give up and have another.”}};
			\node(auto4)[autopilot,action,right=of auto3,yshift=0em]{\gettext{Take/use/drink more and eventually fall asleep}};
			\node(auto5)[autopilot,thought,above=of auto4,callout relative pointer={(0,-1)}]{\gettext{\emph{Next day…\\ — another:}} \\  \gettext{“\textbf{Screw it!}”} \\ \gettext{“I knew I would go back to using. There’s no hope for me.”}};
			\begin{scope}  % [->,color=red!40,line width=1mm]
				\draw[wide autopilot arrow] (initial.south west) -- (auto1.north east);
				\draw[wide autopilot arrow] (auto1) -- (auto2);
				\draw[wide autopilot arrow] (auto2) -- (auto3);
				\draw[wide autopilot arrow] (auto3) to [bend right](auto4.south);
				\draw[wide autopilot arrow] (auto4) -- (auto5);
				\draw[wide autopilot arrow] (auto5) to [bend right] (auto1.east);
			\end{scope}

			\node(mindful)[mindful,draw=none,below right=of top]{\textbf{\textsc{\gettext{Mindful Mode:}}}\\ \gettext{See thoughts as thoughts;\\ Don’t need to act on thoughts}};
			\node(mind1)[mindful,below right=of initial]{\gettext{Pause. Notice your thoughts as just passing thoughts, not truths or commands. \toSelf{This is just a thought. I don’t need to believe it or act on it right now.}}};
			\node(mind2)[mindful,below=of mind1]{\gettext{Focus on your breathing for a few moments to “ground” yourself.}};
			\node(mind3)[mindful,action,below=of mind2]{\gettext{Respond with awareness. Make some tea. Call a supportive friend.}};
			\begin{scope} % [->,ultra thick,color=blue!40,line width=1mm]
				\draw[wide sober arrow] (initial.south east) -- (mind1.north west);
				\draw[wide sober arrow] (mind1) -- (mind2);
				\draw[wide sober arrow] (mind2) -- (mind3);
			\end{scope}
		
			\begin{scope}[->,dashed,line width=1mm,OliveGreen!80]
				\draw (auto3.north east) to [out=85,in=-180,looseness=1.4] 
					(mind1.north west);
				\draw (auto5.north) to [out=90,in=-180]
					node[pos=0.4,solid,draw=none,
						left color=mbrpAuto!10,right color=mbrpSober!20,color=black,opacity=.6,text opacity=1,
						text width=.3\linewidth,
						rotate=20,
						yshift=1em,xshift=0em,
					]{\textsc{\gettext{Never too late to switch to mindful mode.}}}
				(mind1.north west);
			\end{scope}
		\end{tikzpicture}
		\end{adjustbox}


%%
%% 6
%%
\mbrpSession{
	title={\gettext{Surfing the Urge}},
	summary={\gettext{
		\item Urges or cravings to use are like ocean waves. They rise, reach a peak, and eventually pass by.  
		\item Mindfulness helps us slow down and be curious about the experience of an urge/craving. We can pause and ask ourselves \toSelf{Hmm, what does this urge actually feel like right now? In my body? In my mind?}
		\item Mindfulness helps us mindfully “surf” or “ride out” urges, instead of fighting urges or trying to get rid of them. We can practice bringing a sense of openness and acceptance toward the experience of an urge in the moment.  
		\item Mindfulness can also help us be more aware of what we might need and how we can take care of ourselves when we have an urge or craving. We can pause and ask ourselves: \toSelf{What do I really need right now?}
	}},
	formal={\gettext{\item Mindful Check-In; \item Urge Surfing.}},
	informal moments={\gettext{At random times, pause and mindfully check-in with yourself.}},
	informal challenges={\gettext{Practice the \textsc{Sober} Space when challenges or urges come up.}},
	informal activities={\gettext{Try mindful looking. Pause and take in all the sights around you. Notice colors, shapes, lighting, and shadows. Try this when you are outdoors. Try this when you are indoors too.}},
	practice sheets={}
}


%%
%% 7
%%
\mbrpSession{
	title={\gettext{Following Your Values}},
	summary={\gettext{
		\item Personal values are principles and beliefs we have about how we want to live our life and what kind of person we want to be. 
		\item Our values are like a compass or map that guide and direct us in life. Our values can inform the choices we make in daily life – both big and small. 
		\item When we are on automatic pilot, we might act or react in ways not in line with our values. When we are mindful, we can be aware of our actions and act in ways that are in line with our values. 
		\item Recovery is about following our values and finding a sense of meaning and purpose in our lives. Getting in touch with our values can give us the strength to do what is important to us, even when distress and discomfort come up. 
	}},
	formal={\gettext{\item Mindful Check-In;\item Values Meditation.}},
	informal moments={\gettext{At random times, pause and mindfully check-in with yourself.}},
	informal challenges={\gettext{Practice the \textsc{Sober} Space when challenges come up. Consider your values and how you can make a mindful choice in line with your values.}},
	informal activities={\gettext{Practice mindfulness while doing a daily chore, like washing the dishes or folding laundry. Bring a curious awareness to what you are doing in the moment. Connect with your senses (e.g., sights, sounds, touch, smell).}},
	practice sheets={\item \gettext{My Personal Values;}}
}

	%\clearpage
	\subsection{\gettext{My Personal Values}}
		\gettext{Deep down inside, what is important to you? What do you want your life to stand for?}

		\gettext{Personal values are principles and beliefs we have about how we want to live our life and what kind of person we want to be. Values are directions we keep moving in. Values are an ongoing process. For example, if you want to be a loving, caring, supportive partner, that is a value – an ongoing process.}

		% Use this diagram to help you look at your personal values. In each blank circle, fill in a value you hold. You do not have to use every circle, and you may add more circles as needed. For help thinking about your values, take a look at the questions on the next page. 

		\gettext{The following are areas of life that are valued by some people. Not everyone has the same values, and this is not a test to see whether you have the “correct” values. There may be certain areas that you don’t value much; you may skip them if you wish.}

		\bgroup
		\parskip0pt
		\def\valItem#1{\par\vskip.5ex plus.5em\textbf{#1} }
		% \setstretch{.95}
		%\begin{description*}[leftmargin=0pt]
			\gettext{\valItem{Family.} What sort of brother/sister, son/daughter, uncle/aunt, family member do you want to be? What personal qualities would you like to bring to those relationships? What sort of relationships would you like to build? How would you interact with others if you were the ‘ideal you’ in these relationships?}
			\gettext{\valItem{Marriage/couples/intimate relations.} What sort of partner would you like to be in an intimate relationship? What personal qualities would you like to develop? What sort of relationship would you like to build? How would you interact with your partner if you were the ‘ideal you’ in this relationship?}
			\gettext{\valItem{Parenting.} What sort of parent would you like to be? What sort of qualities would you like to have? What sort of relationships would you like to build with your children? How would you behave if you were the ‘ideal you’ as a parent?}
			\gettext{\valItem{Friendships.} What sort of qualities would you like to bring to your friendships? If you could be the best friend possible, how would you behave towards your friends? What friendships would you like to build?}
			\gettext{\valItem{Career/employment.} What do you value in your work? What would make it more meaningful? What kind of worker would you like to be? If you were living up to your own ideal standards, what personal qualities would you like to bring to your work? What sort of work relations would you like to build?}
			\gettext{\valItem{Education/personal growth and development.} What do you value about learning, education, training, or personal growth? What new skills would you like to learn? What knowledge would you like to gain? What further education/learning appeals to you? What sort of student would you like to be? What personal qualities would you like to apply?}
			\gettext{\valItem{Recreation/fun/leisure.} What sorts of hobbies, sports, or leisure activities do you enjoy? How would you like to relax/unwind? How would you like to have fun? What sorts of activities would you like to do?}
			\gettext{\valItem{Spirituality.} Spirituality means different things to everyone. It may be connecting with nature, or it may be participation in an organized religious group. What is important to you in this area of life?}
			\gettext{\valItem{Citizenship/environment/community life.} How would you like to contribute to your community or environment, e.g. through volunteering, or recycling, or supporting a group/charity/cause/political party? What sort of environments would you like to create at home, at work, in your community? What environments would you like to spend more time in?}
			\gettext{\valItem{Health.} What are your values related to maintaining your physical well-being? How do you want to look after your health, with regard to sleep, diet, exercise, smoking, alcohol, etc.? Why is this important?}
		%\end{description*}
		\egroup

	\clearpage
	%\vskip-2mm
	\subsection{\gettext{\textsc{Sober} Space for values}}
		% \vskip-2mm
		\gettext{The \textsc{Sober} Space is an \emph{on-the-go} mindfulness practice that you can do anywhere, anytime because it is brief, simple, and flexible.}

		\gettext{You can use the \textsc{Sober} Space as a way to make mindful choices and follow your values and goals.}

		\begin{itemize}[leftmargin=10mm]
			\itemStop{\gettext{S}}{\gettext{Stop.}} \gettext{Remember to stop or “pause” to do this exercise. This is the first step in stepping out of automatic pilot.}
			\itemStop{\gettext{O}}{\gettext{Observe.}} \gettext{Observe what is going on in the moment, both around you and inside of you (body sensations, emotions, and thoughts). Try to observe with a sense of curiosity and nonjudgment.}
			\itemStop{\gettext{B}}{\gettext{Breathe.}} \gettext{Notice the sensations of the breath in your body as you take a few slow breaths in and out.}
			\itemStop{\gettext{E}}{\gettext{Expand.}} \gettext{Expand your awareness to your body and the situation. \emph{Expand your awareness even more to consider your values.}}
			\itemStop{\gettext{R}}{\gettext{Respond.}} \gettext{Respond to the situation with awareness. \emph{How can your values guide you in responding to the situation?}}
		\end{itemize}
	%\vfill
	\clearpage

	\subsection{\gettext{\textsc{Sober} Space for values: anxiety}}
	\soberSpace{
		title1={\gettext{“Choice Point” Situation}},
		title2={\gettext{Feel anxious before job interview.\\ Not sure whether to go.}},
		autopilot={
			thought/{\gettext{I’m just going to mess up the interview.}},
			thought/{\gettext{What’s the point. I probably won’t even get the job.}},
			action/{\gettext{Feel more anxious and start to also feel sense of shame.}},
			action/{\gettext{Don’t go to job interview. Stay home and lie in bed.}}
		},
		sober S={\gettext{Pause and step out of automatic pilot mode.}},
		sober O={\gettext{{Heart beating fast;},{Anxious;},{Thoughts about avoiding interview.}}},
		sober B={\gettext{Take a few slow, mindful breaths in and out. Focus your attention on the breath.}},
		sober E={\gettext{Expand your awareness back to yourself and the situation. Expand your awareness even more to consider your personal values.}},
		sober R={\gettext{{Say to yourself: \toSelf{Getting a job is important to me right now. I’ll just do the best I can.}},{Go to job interview.}}}
	}
	\clearpage
	\subsection{\gettext{\textsc{Sober} Space for values: boredom}}
	\soberSpace{
		title1={\gettext{“Choice Point” Situation}},
		title2={\gettext{Bored on a Sunday. Not sure what to do with yourself.}},
		autopilot={
			thought/{\gettext{What should I do with myself? Maybe I’ll just chill.}},
			thought/{\gettext{I would feel more relaxed if I could get a little high.}},
			action/{\gettext{Feel craving to get high.}},
			action/{\gettext{Look around house for drugs or alcohol you might still have.}}
		},
		sober S={\gettext{Pause and step out of automatic pilot mode.}},
		sober O={\gettext{{Tension in shoulders;},{Agitated;},{Thinking: \toSelf{What should I do with myself?}}}},
		sober B={\gettext{Take a few slow, mindful breaths in and out. Focus your attention on the breath.}},
		sober E={\gettext{Expand your awareness back to yourself and the situation. Expand your awareness even more to consider your personal values.}},
		sober R={\gettext{{Say to yourself: \toSelf{You know what, I care about my daughter and want to spend more quality time with her.}},{Call daughter and plan to meet for lunch.}}}
	}
	% no own SOBER handouts here?



%%
%% 8
%%
\mbrpSession{
	title={\gettext{Exploring Your Needs}},
	summary={\gettext{
		\item We can turn toward substance use because we are just trying find a way to meet our needs in the moment. These include wholesome and healthy needs that we all have as human beings, like the need to feel safe, to get some relief, to feel in control, connect with other people, or to feel happy and alive.   
		\item Substance use does not actually deliver on their promise. In the long-run, substance use does not fulfill our needs. Of course, it’s completely understandable that we turn towards substance use. As human beings, we all can turn to the wrong things to meet our needs. Having an urge to use substances or seeking out substances is not “bad” in any way. In those moments we are just trying to take care of ourselves and meet our needs, like any other human being.
		\item Mindfulness helps us look “beneath” the initial urge for substances and explore deeper needs we may have in the moment. When we have an urge, we can pause and ask ourselves: \toSelf{what do I really need right now?} Recognizing what we actually need in the moment helps us make wise choices, instead of automatically turning to substances.
	}},
	formal={\gettext{\item Mindful Check-In;\item Exploring Your Needs Meditation.}},
	informal moments={\gettext{At random times, pause and mindfully check-in with yourself.}},
	informal challenges={\gettext{Try doing the \textsc{Sober} Space when challenges or urges come up. Think about your needs. Ask yourself: \toSelf{What do I really need right now?}}},
	informal activities={\gettext{Try practicing mindfulness of nature when you are outside. Bring a curious awareness to different aspects of the natural environment, such as the sky, clouds, the wind, plants, trees, or animals.}},
	practice sheets={}
}

	\pageBasicNeeds




