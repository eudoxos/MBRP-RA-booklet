%%
%% TODO: sync with EN
%%
\bgroup
	\cleardoublepage
	\setcounter{section}{-1}
	\titleformat{\section}[frame]{}{}{1em}{\Large\scshape\bfseries}
	%\titleformat{\section}{}{}{0em}{\Large\scshape\bfseries}
	\section{Startovací balíček}
\egroup	

% \renewcommand{\thesection}{S}

% \emph{TODO: Startovací balíček existuje primárně jako webová stránka, \url{http://brzdime.cz/start}. Její obsah se od původního startovacího balíčku zde již poněkud odchyluje. Pokud bude i PDF verze, je třeba ho směrem sem synchronizovat.}

\subsection{Program MBRP}
	Všímavost znamená: \textbf{být vědomě s právě probíhajícím vlastním prožíváním} (například jak se cítíme a co vnímáme okolo sebe) \textbf{s otevřeností a bez hodnocení}.

	Program MBRP (\emph{Mindfulness-Based Relapse Prevention}, Prevence relapsu pomocí všímavosti) je doléčovací a preventivní program pro závislosti — v širokém smyslu (opakované chování s nepříznivými důsledky), který zahrnuje
	\begin{itemize*}
		\item látkové závislosti (alkohol, drogy, tabák, léky);
		\item závislostní chování:
			\begin{itemize*}
				\item vnější: jídlo, práce, sport, sex, pornografie, hazardní hry, sebepoškozování, nakupování, …;
				\item vnitřní: častá sebekritika či devalvace ostatních, útočnost, úzkostnost, vyhýbavost, submisivita, …
			\end{itemize*}
	\end{itemize*}

	MBRP je strukturovaný trénink dovednosti být všímavý — tedy orientovaný ve svém současném prožívání jak se děje — pomocí různých vedených cvičení. Lepší orientace ve svém prožívání snižuje pravděpodobnost opětovného upadnutí do závislostního chování (relapsu).

	Program má tři pilíře, hlavní součásti — všechny jsou doporučené, ale též dobrovolné:
	\begin{description*}
		\item[formální cvičení všímavosti:] pravidelné (např. každodenní) puštění si nahrávek a cvičení s nimi, ve vlastním vyčleněném čase;
		\item[neformální cvičení:] přinášení všímavosti do činností a situací během vašeho každodenního života;
		\item[setkání skupiny,] sloužící ke strukturování cvičení, sdílení zkušeností se cvičením a další podpoře.
	\end{description*}

	MBRP vám může pomoci:
	\begin{itemize*}
		\item Lépe si uvědomovat spouštěče relapsu.
		\item Zvládat spouštěče aktivně — nejen na ně reagovat.
		\item Pracovat s obtížnými pocity zdravým způsobem.
		\item Být k sobě laskavější.
		\item Kultivovat životní styl, který podporuje dlouhodobé uzdravení.
	\end{itemize*}


	\subsubsection*{Výzkum všímavosti a MBRP}
		Výzkum ukazuje, že učení se a cvičení všímavosti:
			\begin{itemize*}
				\item[⇩] Snižuje riziko relapsu (= upadnutí do opakování návyku).
				\item[⇩] Snižuje nutkání/bažení po návyku.
				\item[⇧] Zlepšuje emoční spokojenost.
			\end{itemize*}
		Výsledky výzkumu MBRP jsou podrobněji popsané v článcích odkázaných na stránce \href{https://brzdime.cz/mbrp/\#literatura}{MBRP / Literatura}\footnote{\href{https://brzdime.cz/mbrp/\#literatura}{brzdime.cz/mbrp/\#literatura}}.
		Zde jsou výsledky výzkumu programu MBRP:
		\begin{enumerate}
			\item Bowen, S., Witkiewitz, K., Clifasefi, S. L., Grow, J., Chawla, N., Hsu, S. H., \& Larimer, M. E. (2014). \emph{Relative efficacy of mindfulness-based relapse prevention, standard relapse prevention, and treatment as usual for substance use disorders: a randomized clinical trial.} JAMA Psychiatry, 71(5), 547-556.
			\item Li, W., Howard, M. O., Garland, E. L., McGovern, P., \& Lazar, M. (2017). \emph{Mindfulness treatment for substance misuse: A systematic review and meta-analysis.} Journal of Substance Abuse Treatment, 75, 62-96.
		\end{enumerate}


\subsection{Formální cvičení}
		Nahraných cvičení je celkem 12. Na každém setkání skupiny doporučíme cvičení do dalšího týdne. Všechny nahrávky včetně popisků najdete na \href{https://brzdime.cz}{brzdime.cz}, kde si je můžete i pustit (z telefonu či počítače). Nahrávky si můžete i stáhnout a přehrát jiným způsobem, který Vám více vyhovuje (offline z telefonu, MP3 přehrávač, …).

		Je dobré mít ve cvičení pravidelnost, rutinu. Může to vypadat například takto:
		\begin{description*}
			\item[Honza:] \emph{Přibližně 6 dní v týdnu poslouchám 5–10 minut nahraných cvičení. Cvičím před snídaní. V sedě na židli. Cvičit každý den ve stejný čas mi pomáhá na cvičení nezapomenout. Otevřu si brzdime.cz na telefonu a tam si nahrávky pouštím. Cvičení všímavosti je součást mého života, protože mi pomáhá více si uvědomovat, co se děje, a díky tomu se mohu lépe udržet na cestě zotavení.}
			\item[Alena:] \emph{Poslouchám 10- či 15-minutová cvičení všímavosti asi tak 5 dní v týdnu. Cvičím večer před spaním, když už jsem po sprše v pyžamu. Sedím na posteli a polštáři si vypodložím záda. Nahrávky si pouštím na MP3 přehrávači, který jsem si půjčila od svého terapeuta. Mám sluchátka, aby mě méně rušil hluk zvenku. Když se držím pravidelného cvičení, cítím se více ukotvená a lépe zvládám stres, který se v mém životě objeví.}
		\end{description*}
	\subsection{Neformální cvičení}
		Neformální cvičení využijete „za pochodu“, kdykoliv během všedního dne.
		\begin{description*}
			\item[Okamžiky:] náhodně se během dne zastavte a
				\begin{itemize*}
					\item zpomalte se několika vědomými dechy;
					\item dejte si 1 minutu, abyste se na sebe napojili („Jak se právě teď cítím?“).
				\end{itemize*}
			\item[Obtíže:] když se objeví náročná situace, např. hádka, těžké pocity, nutkání k užití…:
				\begin{itemize*}
					\item projděte situací všímavě, méně reaktivně;
					\item použijte \textsc{Brzd}u.
				\end{itemize*}
			\item[Činnosti:] při běžných činnostech, např. jedení, domácí práce, sprchování se, čištění zubů, chůze někam…:
				\begin{itemize*}
					\item vneste do přítomného okamžiku zvídavou pozornost;
					\item napojte se na pět smyslů (zrak, sluch, chuť, čich, hmat).
				\end{itemize*}
		\end{description*}
	\subsubsection*{\textsc{Brzda}}
		Krátké, jednoduché a velmi přizpůsobivé cvičení všímavosti za pochodu, které můžete použít kdekoliv a kdykoliv. Podívejte se na str.~\pageref{sober}, kde je podrobně popsané.
\subsection{Obvyklé potíže}
	%{\setstretch{1.01}
	Je zcela v pořádku, normální a běžné zakoušet při cvičení obtíže, které jsou níže rozebrané — i nějaké jiné. Není na nich nic „špatného“ či „vadného“. Nejsou poruchou procesu, naopak, jsou jeho nedílnou součástí. \emph{Můžete} jimi projít a ve cvičení pokračovat. Zde jsou tipy, jak se skrz různé obtíže propracovat, pokud se objeví.
	\begin{description*}
		\item[„Moje mysl se neustále toulá.“] Je naprosto normální a běžné zakoušet toulavou mysl a mít během cvičení všímavosti spoustu myšlenek. Je to jedna z věcí, kterou mysl dělává: toulá se. Nemusíte se snažit myšlenky zastavit či je potlačit. Nakolik to jde, jen si uvědomte probíhající zkušenost, s určitou zvědavostí a bez odsuzování (\toSelf{Hele, podívej se, pozornost už zase odešla}). Některé myšlenky mohou být až znepokojivě „zrychlené“ či „neodbytné“. V těchto chvílích zkuste prozkoumat, jaké je se i k této zkušenosti postavit s přijetím a laskavostí (\toSelf{Jaké by to asi bylo, kdybych ty myšlenky jen zaznamenala a nechala je, místo zápasení s nimi a kritizování se za ně?}).
		\item[„Cítím se nepokojně a nedokážu v klidu sedět.“] Je běžné a zcela v pořádku se při cvičení cítit nepokojně. Může jít o tělesnou neposednost, mentální neklid, či oboje současně. Zkuste, co se stane, když i ten neklid zaznamenáte se zvídavostí (\toSelf{Hmm, jak to vlastně prožívám?}), namísto odsuzování sebe či snahy nepokoj zahnat. Můžete do svého cvičení, vedle těch, která jsou vsedu, přidat všímavou chůzi a všímavý pohyb.
		\item[„Jsem ospalá a při cvičení usínám.“] Usnout při cvičení je v pořádku. Netřeba se toho bát! Ospalost se objevuje víc při poloze vleže. Zkuste místo toho cvičit ve vzpřímeném sedu, případně mít pootevřené či otevřené oči (s pohledem uvolněně sklopeným šikmo dolů před sebe).
		\item[„Nezvládám to“ či „Nedělám to dobře“.] Myšlenky tohoto typu jsou zcela normální — začináme pochybovat o sobě a své schopnosti všímavost cvičit. Zkuste zlehka tyto myšlenky zaznamenat, se zvídavostí a laskavostí (\toSelf{Helemese, už zase přišlo tohle pochybování.}). Připomínejte si, že není nic takového jako cvičit „správně“ či „špatně“. Je jen provádění cvičení a zjistit, co přijde. Pokud se vaše mysl hodně toulá či je pro vás těžké zaměřit pozornost, je to zcela v pořádku a normální.
		\item[„Necítím se o nic lépe. Co je to se mnou?“] Při cvičení je běžné cítit touhu, abychom se cítili lépe, či se soudit za to, že se cítíme tak, jak se cítíme. Zkuste se na chviličku zastavit a vzít na vědomí tuto touhu se cítit jinak (\toSelf{Páni, teď vidím, jak sama sebe dostávám pod tlak, když se chci cítit jinak, než se cítím.}). Jaké by bylo dovolit si cítit se v tuto chvíli právě tak, jak se cítíte? Cvičení všímavosti zahrnuje kultivaci přijímání k naší prožívané zkušenosti — nejde o vytváření tlaku na sebe, abychom se cítili jinak či lépe.
		\item[„Pořád na cvičení zapomínám.“] Je úplně normální a v pořádku na cvičení zapomenout či mít období, kdy jsme zaneprázdněni či nás něco od cvičení odtáhne. To se děje nám všem. I pokud jste zmeškali něklik dní či týdnů cvičení, kdykoliv se můžete ke cvičení vrátit a pokračovat. Důležité je mít vnitřní závazek k pravidelnému cvičení v dlouhém časovém horizontu a ke cvičení se vracet znova a znova. Připomeňte si, že na cvičení všímavosti nemůsíte ze svého dne ukrojit obrovské porce času. I kdyby to bylo 5 či 10 minut denně, dlouhodobě to přinese užitek. Dělat malé cvičení každý den je jako každý den mozek „dobíjet“, aby zůstal silný a zdravý. Můžete si také nastavit alarm na telefonu, aby vám cvičení připomenul.
	\end{description*}

\subsubsection*{Proč u toho vydržet?}
	\begin{description*}
		\item[Mozek doslova roste a vytváří nová spojení po celý život,] včetně dospělosti. Nenarodíme se jednoduše s mozkem, který nám už pak na celý život zůstane. Náš mozek se neustále proměňuje podle zkušeností, které prožíváme, a informací a dovedností, které se učíme a cvičíme. Kdykoliv se učíme či cvičíme něco nového, např. nějakou dovednost, vytvářejí se v mozku nové spoje. S dalším a dalším cvičením té dovednosti se tyto spoje dále zesilují. Neuroplasticita (dosl. nervová tvárnost) je odborné slovo, kterým se tyto změny v mozku popisují.
		\item[Mozek je jako sval, který je možné posílit úsilím a cvičením.] Podobně jako fyzická cvičení posilují tělo, cvičení mozku mentálními cvičeními — jako je např. všímavost — posiluje váš mozek. Výzkum ukazuje, že pravidelné cvičení všímavosti mozek pozorovatelně mění a posiluje oblasti mozku, které se podílejí na zvládání stresu a emocí. Tyto studie též ukázaly, že hladina stresu se pravidelným cvičením všímavosti snižuje.
		\item[Zapamatujte si:] máte schopnost změnit svůj mozek. Všímavost je vědou podepřený nástroj, který můžete používat ke změně mozku a zvýšení schopnosti zvládat náročné situace.
	\end{description*}


\endinput


	\clearpage
	\subsection{Vlastní plán \normalPencilLeftDown}
		\paragraph{Počet dní v týdnu, kdy budu cvičit s nahrávkou:} \Blank{1}\par Doporučujeme si pustit nahrané cvičení denně či obden (4–6 dní v týdnů). Pravidelné cvičení udrží váš „sval všímavosti“ v kondici.
		\paragraph{Denní doba, kdy budu obvykle cvičit:} \Blank{2} \par Zde je pár příkladů denní doby, která se používá:
			\begin{enumerate*}
				\item Ráno: před snídaní, po sprše, při přípravě na den.
				\item Večer: po převlečení do pyžama, během večer ukládací rutiny.
				\item Přes den: během přestávky na oběd.
			\end{enumerate*}
		\paragraph{Jak si budu pouštět nahrávky? [?]} \Blank{2}
		\paragraph{Kde budu cvičit?} \Blank{2}\par Zde je pár možností: ložnice, v autobuse či vlaku (na sluchátka), v autě (zaparkovaném), v kanceláři, venku.
		\paragraph{V jaké pozici?} \Blank{2} \par Doporučujeme sedět vzpřímeně, neopírat se. Podle potřeby použijte polštářky jako oporu pro záda.
			\begin{enumerate*}
				\item V sedě na židli, chodidla na zemi.
				\item V sedě na pohovce, nohy zkřížené či na zemi.
				\item V sedě na posteli, zkřížené nohy.
				\item V sedě na zemi (je možné se opřít zády o zeď).
			\end{enumerate*}
			Lze cvičit i v leže. V této pozici je však snažší upadnout do ospalosti. Pokud byste se v leže cítili velmi ospalí, zkuste jednu z uvedených vzpřímených pozic.
	\clearpage
	\subsection{Všímavost za pochodu, během všedního dne}
		\begin{description}
			\item[\mbrpIcon{pause} Všímavé zastavení.] Náhodně během dne se \emph{zastavte} a 
				\begin{itemize*}
					\item zpomalte se několika vědomými dechy;
					\item dejte si 1 minutu, abyste se na sebe napojili (Jak se právě teď cítím?).
				\end{itemize*}
			\item[\mbrpIcon{head-brain} Všímavé zvládání obtíží.]
				Když se objeví náročná situace, např. hádka, obtížné pocity, nutkání k užití…:
				\begin{itemize*}
					\item projděte situací všímavě, méně reaktivně;
					\item použijte \textsc{Brzd}u.
				\end{itemize*}
			\item[\mbrpIcon{apple} Všímavé činnosti.]
				Když se věnujete běžným činnostem, jako je např. jedení, domácí práce, sprchování se, čištění zubů, chůze někam…:
					\begin{itemize*}
						\item vneste do přítomného okamžiku zvídavou pozornost;
						\item napojte se na 5 smyslů (zrak, sluch, chuť, čich, hmat).
					\end{itemize*}
		\end{description}
		\vfill

	\pageSOBER
	\pagePracticeLog

	\clearpage
	\subsection{Proč u toho vydržet?}

		\mbrpIcon{head-brain} Myslete na svůj MOZEK! 

		\textbf{Mozek doslova rostoe a vytváří nová spojení po celý život}, včetně dospělosti. Nenarodíme se jednoduše s mozkem, který nám už pak na celý život zůstane. Náš mozek se neustále proměňuje podle zkušeností, které prožíváme, a informací a dovedností, které se učíme a cvičíme. Kdykoliv se učíme či cvičíme něco nového, např. nějakou dovednost, vytvářejí se v mozku nové spoje. S dalším a dalším cvičením té dovednosti se tyto spoje dále zesilují. \emph{Neuroplasticita} (dosl. nervová tvárnost) je odborné slovo, kterým se tyto změny v mozku popisují.

		\mbrpIcon{lifting-weight} \mbrpIcon{biceps-flex}
		\textbf{Mozek je jako sval, který je možné posílit úsilím a cvičením.} Podobně jako fyzická cvičení posilují tělo, cvičení mozku mentálními cvičeními — jako je např. všímavost — posiluje váš mozek. Výzkum ukazuje, že pravidelné cvičení všímavosti mozek pozorovatelně mění a posiluje oblasti mozku, které se podílejí na zvládání stresu a emocí. Tyto studie též ukázaly, že hladina stresu se pravidelným cvičením všíavosti snižuje.

		
		\mbrpIcon{trend-up}
		\textbf{Zapamatujte si: máte schopnost změnit svůj mozek. Všímavost je vědou podepřený nástroj, který můžete používat ke změně mozku a zvýšení schopnosti zvládat náročné situace.}


