\appendix
\cleardoublepage


\def\practiceSheetTitle#1{\clearpage\fancyhead[L]{\large\textbf{#1 \normalPencilLeftDown}}}
\pagestyle{practice-sheet}
\pagePracticeLog
%% 1
\practiceSheetTitle{\circledI{} \textsc{Brzda} s pocitem}
\pageSOBERown{Popište situaci, která vyvolá automatickou reakci.}{Situace}
% Najděte ještě jeden příklad. Popište jinou situaci, která vyvolá automatickou reakci.
%% 4
\practiceSheetTitle{\circledIV{} \textsc{Brzda} na spouštěč}
\pageSOBERown{Zamyslete se nad \emph{spouštěčem}, který přináší nutkání či bažení po alkoholu či drogách.}{{\Large\textup{\lightning}} Primární spouštěč}
% Najděte ještě jeden příklad \emph{spouštěče}, který přináší nutkání či bažení po alkoholu či drogách.
%% 7
% \pageSOBERown{\textsc{Brzda} na spouštěč: vlastní}{} Primární spouštěč}
% \clearpage
% \subsection*{\circled{7} Mé osobní hodnoty \normalPencilLeftDown}
\practiceSheetTitle{\circledVII{} Mé osobní hodnoty}
	Co je pro vás, úplně uvnitř a niterně, důležité? Na čem byste chtěli, aby váš život stál? Do políček „pavouka“ vepište své hodnoty. Nemusíte využít všechna políčka, a můžete si přikreslit další dle potřeby.

	% Osobní hodnoty jsou principy a přesvědčení o tom, jak chceme žít a jací chceme být. Hodnoty jsou směry, ve kterých se pohybujeme. Například, pokud chcete být milujícím, laskavým a podpůrným partnerem, to je hodnota — pokračující proces.
	% \emph{Jako osnovu pro reflexi o svých hodnotách můžete využít text.}

		\begin{center}
		\begin{tikzpicture}
			% https://tex.stackexchange.com/a/145381/6758
			\draw (0:0) node[circle,draw,inner sep=.05\linewidth](self){\textbf{já}};
			\foreach \a in {1,2,...,6}{
				\draw (13+\a*360/6: .35\linewidth) node[circle,draw,inner sep=.09\linewidth](circ){ };
				\draw (self)--(circ);
			}
		\end{tikzpicture}
		\end{center}

%%
%% Starter pack
%%
\practiceSheetTitle{\textcircled{\scriptsize\normalfont\sffamily S} Motivace ke cvičení}
	Jaké jsou vaše osobní motivace ke cvičení všímavosti? Zaškrtněte některé z uvedených možností, případně dopište své vlastní osobní motivy.
	\begin{itemize}
		\itemSq Zotavení ze závislosti je pro mě důležité.
		\itemSq Péče o sebe je důležitou součástí mého života.
		\itemSq Chci posílit svou schopnost zvládání stresu v životě.
		\itemSq Rád zkouším nové věci.
		\itemSq Je pro mě důležité, že všímavost je podepřená výzkumem.
		\itemSq Uvědomuju si, že ostatím všímavost v zotavení pomohla.
		\itemSq Chci se naučit účinné strategie zvládání životních situací.
		\itemSq Svého mentálního zdraví si cením minimálně tak jako tělesného.
		\itemSq Chci žít zdravý a vyvážený život.
		\itemSq \Blank{1}
		\itemSq \Blank{1}
		\itemSq \Blank{1}
	\end{itemize}


\practiceSheetTitle{\textcircled{\scriptsize\normalfont\sffamily S} Plán cvičení}
		\textbf{Počet dní v týdnu, kdy budu cvičit s nahrávkou:} \Blank{1}\par Doporučujeme si pustit nahrané cvičení denně či obden (4–6 dní v týdnů). Pravidelné cvičení udrží váš „sval všímavosti“ v kondici.

		\textbf{Denní doba, kdy budu obvykle cvičit:} \Blank{3} \par Zde je pár příkladů denní doby, která se používá:
			\begin{itemize*}
				\item Ráno: před snídaní, po sprše, při přípravě na den.
				\item Večer: po převlečení do pyžama, během večer ukládací rutiny.
				\item Přes den: během přestávky na oběd.
			\end{itemize*}

		\textbf{Jak si budu pouštět nahrávky? [?]} \Blank{2}

		\textbf{Kde budu cvičit?} \Blank{2}\par Zde je pár možností: ložnice, v autobuse či vlaku (na sluchátka), v autě (zaparkovaném), v kanceláři, venku.

		\textbf{V jaké pozici?} \Blank{2} \par Doporučujeme sedět vzpřímeně, neopírat se. Podle potřeby použijte polštářky jako oporu pro záda.
			\begin{enumerate*}
				\item V sedě na židli, chodidla na zemi.
				\item V sedě na pohovce, nohy zkřížené či na zemi.
				\item V sedě na posteli, zkřížené nohy.
				\item V sedě na zemi (je možné se opřít zády o zeď).
			\end{enumerate*}
			Lze cvičit i v leže. V této pozici je však snažší upadnout do ospalosti. Pokud byste se v leže cítili velmi ospalí, zkuste jednu z uvedených vzpřímených pozic.

